\documentclass{assignment}
\usepackage{mymath}

\no{6}
\class{Real Analysis}

\begin{document}
\maketitle

\section*{Other Problems}
\begin{question}[1]
  Identify the completions of the following metric spaces in simple terms. You need only give an
answer; no proof is necessary.
\begin{qparts}
  \item an open interval $(a,b) \subset \R$, where $a,b$ need not be finite.
  \item a half-open interval $(a,b] \subset \R$
  \item an open ball $B_r(p) \subset \E^n$
  \item a deleted open ball $B'_r(p)$ for some $p\in\E^k, r > 0$
  \item a closed ball $\conj{B_r(p)} \subset \E^n$
  \item the set of rational points $\{(x,y) \in \E^2 \mid x,y \in \Q \}$
  \item the set of integral points $\{(x,y) \in \E^2 \mid x,y \in \Z \}$
  \item a complete metric space $(X, d)$
  \item a compact metric space $(X, d)$ 
\end{qparts} 
\end{question}
\begin{proof}\leavevmode
\begin{qparts}
  \item $[a,b]$ if neither $a$ nor $b$ is infinite, else open the interval for whichever ``endpoint''
is not finite.
  \item $[a,b]$
  \item $\conj{B_r(p)}$
  \item $\conj{B_r(p)}$
  \item $\conj{B_r(p)}$
  \item Since the rationals are dense in the reals, we have that the completion is really the whole 
plane $\E^2$. 
  \item Since the integers are not dense in the reals, there is a minimum distance between two distinct
integral points in $\E^2$, and by the main mechanism of the proof from 2 (a), we have that it is complete.
(Note that the method of the proof works even for non-finite metric spaces; really, all we need is a 
lower bound on the distance between distinct points).
  \item $(X,d)$
  \item $(X,d)$
\end{qparts}
\end{proof}

\begin{question}[2]
  \begin{qparts}
    \item Show that every finite metric space is complete.
    \item Let $X$ be any set and $d$ the discrete metric. What are the Cauchy sequences in $X$? Under 
what conditions is $X$ complete?
  \end{qparts}
\end{question}
\begin{proof}\leavevmode
  \begin{qparts}
     \item Suppose $(M, d)$ is a finite metric space. Since it is finite, there is some pair of ``closest
points;'' that is, $$\exists \varepsilon > 0 \st \forall x_1, x_2 \in M, d(x_1, x_2) > \varepsilon.$$ 
Note that this blows up if $|M| = 1$, so suppose WLOG $|M| > 1$; the single-point case is trivial anyway.
Now, let $\{p_n\} \subset M$ be a Cauchy sequence. Then $$\exists N\in\N \st \forall m,n \geq N, 
d(p_n, p_m) < \varepsilon.$$ But we had that no two points can be closer than $\varepsilon$ from before,
so indeed in order for $\{p_n\}$ to be Cauchy, $p_m = p_n$ for $m,n \geq N$. Thus, since $\{p_n\}$ is 
constant in $M$, it converges, and so $(M, d)$ is complete.

    \item Since $x_1, x_2 \in X, x_1 \neq x_2 \Rightarrow d(x_1, x_2) = 1$, we just pick $\varepsilon
< 1$ and thus have that, by a similar argument as in (a), the only Cauchy sequences are those that 
are eventually constant. So then $X$ seems like it's pretty much always complete, since every Cauchy
sequence is eventually constant in $X$ and thus converges in $X$.
  \end{qparts}
\end{proof}

\begin{question}
  For each of the following statements, write its negation. Are the statements in (a) and (b) true?
\begin{qparts}
  \item For every metric space $(X,d)$ there exists a Cauchy sequence $\{p_n\} \subset X$ that converges.
  \item There exists a metric space $(X, d)$ such that every sequence $\{p_n\} \subset X$ either has a 
convergent subsequence or is unbounded. 
  \item Let $f: [0,1]\to\R$. For all $x_0\in[0,1]$ and all $\varepsilon\in\R^{>0}$, there exists
$\delta\in\R^{>0}$ such that if $|x - x_0| < \delta$ then $|f(x) - f(x_0)| < \varepsilon.$
\end{qparts}
\end{question}
\begin{proof}\leavevmode
\begin{qparts}
  \item The negation of this statement is
$$\exists (X,d) \st \forall \{p_n\} \subset X \text{ that are Cauchy, } \{p_n\} \text{ diverges.}$$
In this case, the original statement is true, since we can just take the constant sequence in any
metric space with at least one element, and the degenerate case is trivial.  

  \item The negation of this statement is 
$$\forall (X, d), \exists \{p_n\} \subset X \st \{p_n\} \text{ does not have a convergent subsequence and
is not unbounded.}$$
The predicate is tautological, since bounded $\Rightarrow$ convergent subsequence, and so if the
sequence is unbounded the predicate is true, and if not, then it is bounded, and thus has a convergent subsequence,
and thus the predicate is true. So the original statement is true because the predicate is always true.\\

(NB: Maybe ``predicate'' should be replaced with ``consequent'' above.)
  
  \item The negation of this statement is $$\exists x\in[0,1], \varepsilon\in\R^{>0} \st \forall \delta
    \in\R^{>0}, |x - x_0| < \delta \wedge |f(x) - f(x_0)| \geq \varepsilon.$$ The original statement 
    is just a statement of continuity, so the truth value depends on the function $f$. For example,
    $f(x) = x^2$ is continuous, so the original statement holds, but it's easy to construct a pathological
    function that just takes the values $0$ and $\infty$ in order to falsify the original statement.
\end{qparts} 
\end{proof}


\end{document}
