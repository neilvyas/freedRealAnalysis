\documentclass{assignment}
\usepackage{mymath}

\begin{document}
\header{Real Analysis I}{Homework 3}

Note for the grader: I can do all these problems just fine, I just ran into a severe time crunch this
week because of involvement with student organizations, and my job. Sorry!

\section*{Rudin Problems}
% ch 2: 4,5,7,9,10 (except statement about compactness)
\begin{question}[2]
  A complex number $z$ is said to be \emph{algebraic} if there are integers $a_0,\dots,a_n$, not all
zero, such that $$a_0z^n + a_1z^{n-1} + \dots + a_{n-1}z + a_n = 0.$$ Prove that the set of all 
algebraic numbers is countable. \emph{Hint:} For every positive integer $N$ there are only finitely 
many equations with $$n + |a_0| + |a_1| + \dots + |a_n| = N.$$
\end{question}
\begin{proof}
  Grant the hint; actually, note that this is true because the sum constraint means that none of the
  $a_i$ may be bigger than $N$ in magnitude, and since they have magnitude not all 0, there can't 
  be more than $N$ of them. So we have that $1 \leq n \leq N$ and $0 \leq |a_i| \leq N$. Now we have 
  to see how the hint connects to the problem.
  \end{proof}

\begin{question}[4]
  Is the set of all irrational real numbers, denoted $\R\setminus\Q$, countable?
\end{question}
  Intuitively, it shouldn't be, since the rationals are countable via the argument given in class 
(which was an injection from $\N\times\N\to\N$), so we should be able to find some justification 
along these lines.
\begin{proof}
Suppose not, that is, suppose that $\R \setminus \Q$ is countable. Then consider $\R' = (\R \setminus
\Q) \cup \Q$; note that this is isomorphic to/is $\R$. Since the union of countable sets is countable,
via a result proved in class (intuitively, ``map'' one of the sets to the even numbers, and the other
to the odd), we have that $\R$ is countable, a contradiction. So it must be that $\R\setminus\Q$ is
uncountable.
\end{proof}

\begin{question}[5]
  Construct a bounded set of real numbers with exactly three limit points.
\end{question}
\begin{proof}
  We can just union some disjoint sets that converge to distinct real numbers using the standard method
of constructing such sets; bounding them will also be trivial. Take $$A = \{\frac{1}{n} \mid n\in\N\}
\bigcup \{-2 + \frac{1}{n} \mid n\in\N\} \bigcup \{2 + \frac{1}{n} \mid n\in\N \}.$$ So $A$ is bounded
 by 42 and has the limit points 0, $-1$, and 1.
\end{proof}

\begin{question}[7]
  Let $A_1, A_2, \ldots$ be subsets of a metric space.
\begin{qparts}
\item If $B_n = \bigcup_{i=1}^n A_i,$ prove that $\conj{B} = \bigcup_{i=1}^n \conj{A_i}.$
\item If $B_n = \bigcup_{i=1}^\infty A_i,$ prove that $\conj{B} \supset \bigcup_{i=1}^n \conj{A_i}.$
\end{qparts}
Show, by example, that this inclusion can be proper.
\end{question}
\begin{proof}
  By the definition of closure, we have that $\bigcup A_i \subset \bigcup \conj{A_i}$, since $A_i 
  \subset \conj{A_i}$. So we have that $\conj{B} \subset $
\end{proof}

\begin{question}[9]
Let $E^0$ denote the set of all interior points of a set $E$. 
\begin{qparts}
\item Prove that $E^0$ is always open.
\item Prove that $E$ is open iff $E = E^0$.
\item If $G\subset E$ and $G$ is open, prove that $G\subset E^0$.
\item Prove that the complement of $E^0$ is the closure of the complement of $E$.
\item Do $E$ and $\conj{E}$ always have the same interiors?
\item Do $E$ and $E^0$ always have the same closures?
\end{qparts}
\end{question}
\begin{proof}
\begin{qparts}
\item $E^0$ is open iff every point of $E^0$ is in the interior of $E^0$, so take $p\in E^0$; we want 
to show that $p\in(E^0)^0$. Since we have $p\in E^0$, we have that every point 
\end{qparts}
\end{proof}

\begin{question}[10]
  Let $X$ be an infinite set. For $p, q \in X$, define 
$$d(p,q) = \begin{cases} 1 &p = q \\ 0 &p\neq q.\end{cases}$$
Prove that this is a metric. Which subsets of the resulting metric space are open? Which are closed?
Which are compact?
\end{question}
\begin{proof}
  Note that $\forall p,q\in X, d(p,q) \geq 0$, so we have that $d$ is positive. Note also that $d(p,
q) = 0 \iff p = q,$ which we can obviously see from the definition. It remains only to check that the 
triangle inequality holds, so let $p,q,s\in X$; we want to show that $$d(p,q) \leq d(p,s) + d(s,q).$$
We will show the result by a case analysis.  \\

Let's consider the case where $p=q$. So the LHS is zero, and by positivity we don't even have 
to check the right side, since $d \geq 0$ and the sum of two non-negative real numbers is non-negative.
Now, let's consider the case where $p\neq q$. Then the LHS is 1. On the RHS, if $s$ is distinct from
both $p,q$, we have that the RHS is 2; otherwise if $s$ coincides with one of $p,q$, we have that the
RHS is 1, so we have the desired result. Thus, $d$ satisifies the triangle inequality. \\

For any $p\in X$, we have that $p$ is not a limit point. This is easy to see; pick any $e\in(0,1)$;
we'll pick $1/42$. Then note that there exists a neighborhood $N_{1/42}(p)$ which contains no points
$q\in X$ where $q\neq p$, since $q\neq \iff d(p,q) = 1 > 1/42$. So every subset $S$ of $X$ contains 
no limit points and it is thus closed. So, since $X \setminus S$ is closed, as well, we have that 
$S = X \setminus (X\setminus S)$ is open, since a set is closed iff its complement is open. So by 
this arguement, it must be that in fact every subset $S\subset X$ is open. \\

Any finite subset of $X$ is compact. 
\end{proof}

\section*{Other Problems}
\begin{question}[1]
  Let $S$ be the circle of unit radius in the Euclidean plane: 
$$S = \{(x,y)\in\E^2\mid x^2 + y^2 = 1\}.$$
\begin{qparts}
\item Show that $S$ is uncountable.
\item Define a metric on $S$ which makes it into a metric space. More precisely, define a metric such
that if $U\subset\E^2$ is an open set, then $U\cap S$ is open.
\end{qparts}
\end{question}
\begin{proof}
  \begin{qparts}
  \item We just need to show that any subset of $S$ is uncountable, so take $$S' = \{(x, 0) \mid 0\leq x
    \leq 1 \}.$$ Note that this is now a subset of the real line $\R$, which is uncountable, so this
    subset is uncountable. Thus, since $S'$ is uncountable, we have that $S$ is uncountable.
  \item 
  \end{qparts}
\end{proof}

\begin{question}[2]
  Suppose $X$ is a metric space with metric $d$.
\begin{qparts}
\item Prove that $$d'(x,y) = \frac{d(x,y)}{1 + d(x,y)}, $$for $x,y\in X$ is also a metric on $X$.
\item Show that $d$ and $d'$ are \emph{topologically equivalent}; that is, a set is $d$-open iff it is
$d'$-open.
\item Let $F$ be a finite set and $d$ a metric on $F$. What are the $d$-open sets? How does you answer
depend on the choice of $d$? When are two metrics on $F$ topologically equivalent?
\end{qparts}
\end{question}
\begin{proof}
  \begin{qparts}
    \item We need to show that $d'$ satisifies the three properties of a metric. Obviously, we have that
      $d'$ is positive, since $d$ is positive and the ratio of two positive numbers is positive, and 
      when $d$ is zero, $d'$ is well defined and is zero. To show the second property, we note that 
      $d' = 0 \iff d = 0$, which is sufficient since $d$ has the desired property. Now we just have
      to show that $d'$ satisifies the triangle inequality. \\

      We want to show that $d'(x,z) \leq d'(x,y) + d'(y,z)$, and we know that $d$ satisifies the triangle
      inequality. So, all that remains is working out a bunch of algebra!
      \begin{align*}
        d'(x,y) + d'(y,z) &= \frac{d(x,y)}{1 + d(x,y)} + \frac{d(y,z)}{1 + d(y,z)} \\
                          &= \frac{d(x,y) + 2d(x,y)d(y,z) + d(y,z)}{1 + d(x,y)d(y,z) + d(x,y) + d(y,z)} \\
                          &\geq \frac{d(x,z)}{1 + d(x,y)d(y,z) + d(x,y) + d(y,z)} \\
                          &\geq \frac{d(x,z)}{1 + d(x,z)} \\
                          &= d'(x,z),
      \end{align*}
      as desired.
    \item Suppose we have a set $D$ that is $d$-open. Then for any $x\in D, \exists\varepsilon\in\R\st 
      \forall y \st d(x,y) < \varepsilon, y\in D$. Now, we want to show that this set is also $d'$-open. 
      So, we should show that distances scale well between $d$ and $d'$. That is, 
      \begin{lemma}
        $$d(x,y) < \varepsilon \iff \exists\delta > 0 \st d'(x,y) < \delta.$$
      \end{lemma}
      \begin{proof}
        Suppose that $d(x,y) < \varepsilon$. Then 
        \begin{align*}  
          d'(x,y) &= \frac{d(x,y)}{1 + d(x,y)} \\
                  &< \frac{\varepsilon}{1 + \varepsilon},
        \end{align*}
        so take $\delta > \frac{\varepsilon}{1 + \varepsilon}$.
      \end{proof}
      Now with the lemma, the proof of the result is simple; since $d(x,y) < \varepsilon, d'(x,y) < 
      \delta \Rightarrow $ the $d$-open set is $d'$-open, and vice versa since the lemma has an $\iff.$

    \item I'm not really sure how to answer this, since I think Dr. Freed didn't cover as much as he 
      would have liked before this homework was due.
  \end{qparts}
\end{proof}



\end{document}
