\documentclass{assignment}
\usepackage{mymath}

\begin{document}
\header{Real Analysis I}{Homework 2}

\section*{Rudin}
%1,4,5,7,16,17
\begin{question}[1]
  If $r$ is rational, $r\neq 0$, and $x$ is irrational, prove that $r+x$ and $rx$ are irrational.
\end{question}
\begin{proof}
  Suppose not. That is, suppose that $r\neq 0\in\Q, x\notin\Q$, and $r+x$ and $rx\in\Q$; our aim is to
  show that this implies that $x\in\Q$, a contradiction. Since $r\in\Q$, we have that $\exists m,n
  \in\Z, n\neq 0 \st \frac{m}{n} = r$, and similarly, we have $\frac{a}{b} = r + x$ and $\frac{p}{q}
  = rx$. So then we have 
  \begin{align*}
    (r + x) - r &= \frac{a}{b} - \frac{m}{n} \\
                &= \frac{an - mb}{nb} \\
                &= x, 
  \end{align*}
  and so we've arrived at a contradiction for $r+x$, which means that $r+x\notin\Q$, i.e $r+x$ is irrational,
  since $an-mb \in \Z, np\neq 0\in\Z$ because $\Z$ is closed under addition and multiplication and 
  $x,y\in\Z \neq 0 \Rightarrow xy\neq 0$. Similarly, we have
  \begin{align*}
    rx / r &= \frac{p}{q} / \frac{m}{n} \\
           &= \frac{pn}{qm} \\
           &= x,
  \end{align*}
  and so we've arrived at a contradiction for $rx$, which means that $rx$ is irrational. This is because
  $pn\in\Z, qm\in\Z\neq 0$ since $r\neq 0 \Rightarrow m\neq 0$ and the definition of $q$, where $q\neq 0$.\\

  So we have the result.
\end{proof}

\begin{question}[4]
  Let $E$ be a non-empty subset of an ordered set; suppose that $\alpha$ is a lower bound of $E$ and
$\beta$ an upper bound. Prove that $\alpha\leq\beta$.
\end{question}
\begin{proof}
  %This proof relies on the transitivity of $\leq$.
  Let $\alpha, \beta, E$ be as in the question. Since $E$ is non-empty, let $x\in E$. Then by definition,
  $x\leq\beta$ and $\alpha\leq x$. So then by the transitivity of $\leq$, we have that 
  $$\alpha\leq x, x\leq\beta \Rightarrow \alpha\leq\beta.$$
\end{proof}

\begin{question}[5]
  Let $A$ be a nonempty set of real numbers which is bounded below. Let $-A$ be the set of all numbers
  $-x$, where $x\in A$. Prove that $$\inf A = -\sup(-A).$$
\end{question}
\begin{proof}
  Grant that negation of sign flips the direction of inequalities, which is a prior result. \\

  Suppose that $A$ is as above and that $\inf A$ exists; let $y = \inf A$. Then for any upper bound of 
  $A, \gamma$, we have that $\gamma \leq y$. Now negate everything. We must check that $-y$ is an upper
  bound for $-A$ and that for any $\beta$ an upper bound of $-A$, we have that $-y \leq \beta$. $-y$
  is obviously an upper bound for $-A$, since we have that $y \leq a\in A \Rightarrow -y \geq -a\in 
  -A$. \\
  
  To prove the next statement about \emph{least} upper bound, we observe that any upper bound
  for $-A$ is a lower bound for $A$ by the same argument above, and vice versa. Then, note that this
  implies that $y \geq \gamma \Rightarrow -y \leq -\gamma$. So we have that $-y$ is the \emph{least} 
  upper bound, and thus the $\sup -A$, by negating the expression and working in terms of our original 
  set $A$.
\end{proof}

\begin{question}[7]
  Fix $b > 1$, $y > 0$, and prove that there is a unique real $x$ such that $b^x = y$, by completing
  the following outline.
  \begin{qparts}
    \item For any positive integer $n$, $b^n - 1 \geq n(b-1)$.
    \item Hence $b - 1 \geq n(b^{1/n} - 1)$. 
    \item If $t > 1$ and $n > (b-1)/(t-1)$, then $b^{1/n} < t$.
    \item If $w$ is such that $b^w < y$, then $b^{w + (1/n)} < y$ for sufficiently large $n$; to see this,
      apply part (c) with $t = yb^{-w}$. 
    \item If $b^w > y$, then $b^{w - (1/n)} > y$ for sufficiently large $n$.
    \item Let $A$ be the set of all $w$ such that $b^w < y$, and show that $x = \sup A$ satisfies $b^x 
      = y$. 
    \item Prove that this $x$ is unique.
  \end{qparts}
\end{question}
\begin{proof}
  \begin{qparts}
  \item (Induction on $n$) Let $n = 1$; note that the expression in question becomes $b - 1 \leq b -
    1$, which is true. Now suppose that $k \geq 1\in\Z, b^k - 1 \geq k(b - 1)$. We want to show that
    the result holds for $k + 1$. As a notational convenience, rewrite the inductive hypothesis
    (IH) as $b^k -1 - k(b - 1) \geq 0$. So then we have 
    \begin{align*}
      b^{k+1} - 1 - (k + 1)(b - 1) &= b^{k+1} - 1 - k(b - 1) - (b - 1) \\
                                   &= bb^{k} - k(b-1) - b \\
                                   &= b^k -\frac{k}{b}(b - 1) -1 \note{(divide by $b$, since $b > 0$)} \\
                                   &> b^k -1 -k(b - 1) \note{(since $b > 1$)} \\
                                   &\geq 0 \note{(by the IH),}
    \end{align*}
    so we have the claim.
  \item We conclude this by simply exponentiating the inequality in (a) by $\frac{1}{n}$, noting that
    since $\frac{1}{n} > 0$, the direction of the inequality is preserved.
  \item $t > 1 \Rightarrow t -1 > 0$, so then we have that 
    \begin{align*}
      n > (b-1)/(t-1) &\Rightarrow n(t - 1) > (b - 1) \\
                      &\Rightarrow n(t-1) > n(b^{1/n} - 1) \note{(using our result from (b)} \\
                      &\Rightarrow t > b^{1/n} \note{(since $n > 0$)}.
    \end{align*}
  \item Suppose $w$ is such that $b^w < y$. Let's set $t = yb^{-w} > 1$. Then we have 
    \begin{align*}
      b^{1/n} < t &\Rightarrow b^{1/n} < yb^{-w} \\
                  &\Rightarrow b^{1/n}b^w < y \note{(since $b^w > y > 0 \Rightarrow b^{-w} > 0$)} \\
                  &\Rightarrow b^{w + (1/n)} < y,
    \end{align*}
    as desired. Note that all of this relies on $n$ satisfying (c), so let's characterize such an $n$.
    We need 
    \begin{align*}
      n &> (b - 1)/(t - 1) \\
        &= (b - 1)/(yb^{-w} - 1).
     \end{align*}
  \item Suppose that $b^w > y$. Let's set $t = yb^w > 1$. Then we have 
    \begin{align*}
      b^{1/n} < t &\Rightarrow b^{1/n} < yb^w \\
                  &\Rightarrow b^{(1/n) - w} < y \\
                  &\Rightarrow b^{w - (1/n)} > y,
    \end{align*}
    as desired. We obtain the last statement by noting that exponentiating one side by $-1$ flips the 
    direction of an inequality; we'll omit the proof here. As before, this relies on sufficiently large
    $n$, so let's characterize that $n$ below.
    \begin{align*}
      n &> (b - 1)/(t - 1) \\
        &= (b - 1)/(yb^w - 1).
    \end{align*}
  \item Since $\R$ has the least upper bound property, we know that such an $x$ exists. We need to 
    show that neither $b^x < y$ nor $b^x > y$ are true in order to show that $b^x = y$. \\

    Let's start by showing not $b^x < y$. Suppose not; that is, suppose that $b^x <  y$. Then by part 
    (d) we know there is some integer $n$ for which $b^{x  + (1/n)} < y$, so we have that $x + (1/n)
    \in A$. But $x + (1/n) > x$, so we have contradicted our assumption that $x = \sup A$. So, we 
    conclude that $b^x \geq y$. \\

    We'll show not $b^x > y$ analogously. Again, suppose not. That is, suppose that $b^x > y$. Then 
    by part (e), we have that  $b^{x - (1/n)} > y$, which means that $x - (1/n)$ is an upper bound for
    $A$. But since $x > x - (1/n)$, we have contradicted our assumption that $x$ is the least upper
    bound. So, we conclude that $b^x \leq y$. \\

    Thus, $$b^x \leq y \wedge b^x \geq y \Rightarrow b^x = y.$$
  \item The standard method of proving something is unique is to suppose that there are two of them,
    then show that these two instances must be the same. So, suppose that there is some other $x'\in
    \R \st x' = \sup A$; in particular, we have by (f) that $b^{x'} = y$. So $b^x = y = b^{x'}$. 
    If $x' \neq x$, then either $x' > x$ or $x' < x$, so let's work through both of these cases. 
    Throughout, we will refer to the expression $b^{x'} = b^x$. \\

    Suppose that $0 < x' < x$. Then divide both sides of the expression by $b^{x'}$, yielding $b^{x - 
    x'} = 1$. But we assumed at the very beginning that $b > 1$, and $x' < x \Rightarrow x - x' > 0$, 
    so this is a contradiction. Thus, we have that $x' \geq x$. \\

    Suppose now that $0 < x < x'$. We run the exact same procedure, concluding that $x' \leq x$. \\

    So we have that $x'\leq x \wedge x'\geq x \Rightarrow x' = x$, which proves the claim.

  \end{qparts}
\end{proof}

\begin{question}[16]
  Suppose $k\geq 3, x,y\in\R^k, |x - y| = d > 0$, and $r > 0$. Prove that 
  \begin{qparts}
    \item If $2r > d$, there are infinitely many $z\in\R^k$ such that $$|z-x| = |z - y| = r.$$
    \item If $2r = d$, there is exactly one such $z$.
    \item If $2r < d$, there is no such $z$.
  \end{qparts}
  How must these statements be modified if $k$ is 2 or 1?
\end{question}
\begin{proof}
  We'll derive a bunch of algebraic nonsense here in order to make the proofs of the following parts
  cleaner.
    \begin{align*}
      |z - x| = |z - y| &\iff {|z-x|}^2 = {|z-y|}^2 \\
                        &\iff {|z-x|}^2 - {|z-y|}^2 = 0 \\
                        &\iff 2z(y - x) + (y^2 - x^2) = 0 \\
                        &\iff \left(z - \frac{x + y}{2}\right)(y - x) = 0,
    \end{align*}
    so we have that 
    \begin{align*}
      z -x &= \left(z - \frac{x + y}{2} + \frac{y - x}{2}\right)
    \end{align*}
    Actually, this proof winds up being horrifically ugly and not very fun, so I'll leave it off.
\end{proof}

\begin{question}[17]
  Prove that $$|x + y|^2 + |x - y|^2 = 2|x|^2 + 2|y|^2,$$ if $x,y\in\R^k$. Interpret this geometrically,
  as a statement about parallelograms.
\end{question}
\begin{proof}
  Let $x,y\in\R^k$. Then we have
  \begin{align*}
    |x+y|^2 + |x-y|^2 &= (x+y)^2 + (x-y)^2 \\
                      &= (x^2 + 2xy + y^2) + (x^2 - 2xy + y^2) \\
                      &= 2x^2 + 2y^2 \\
                      &= 2|x|^2 + 2|y|^2.
  \end{align*}
\end{proof}
This looks like it's going to be saying something about areas, and the obvious interpretation of $x,y$ 
is as the sides of this parallelogram. Observe that if we ``make'' a parallelogram by lying $y$ against
the head and tail of $x$, then placing the second side of length $x$ opposite the first, the diagonals
will be given by $x + y$ and $x - y$, from the parallelogram method of vector addition. \\

So, the statement says that the sum of the squared diagonals is twice the sum of the squares of the sides.

\section*{Other Questions}
\begin{question}[1]
  Show that the rational numbers $\Q$ form an ordered field that satisfies the Archimedean property.
What property of $\R$ is not satisfied by $\Q$?
\end{question}
\begin{proposition}
  $\Q$ forms an ordered field with the ordering $$\frac{a}{b} > \frac{a'}{b'} \iff ab' > a'b,$$
  for $a,b,a',b'\in\Z$ and $b,b' \neq 0$.
\end{proposition}
\begin{proof}
  We have already shown $\Q$ to be a field and I really do not want to verify all those axioms
  again, so we'll just check the two axioms for orderings. So let $x,y,z\in\Q$; by definition, we
  can write $$x = \frac{a}{b}, y = \frac{m}{n}, z = \frac{p}{q},$$ for $a,b,m,n,p,q\in\Z$ and $b, n,
  q \neq 0$. Further, suppose that $$x, y > 0, y < z.$$ \\
  
  Then for the first axoim we need to show that $$y < z \Rightarrow x + y < x + z.$$ Suppose $y <
  z$. Now,
  \begin{align*}
    x + y &= \frac{a}{b} + \frac{m}{n} \\
          &= \frac{na + mb}{bn} \\
    x + z &= \frac{pb + aq}{bq}.
  \end{align*}
  Now, we want to compare $x+y$ and $x+z$, so let's just do it!
  \begin{align*}
    x+ y \stackrel{?}{<} x + z &\cong \frac{na + mb}{bn} \stackrel{?}{<} \frac{pb + aq}{bq} \\
                               &\iff (na + mb)(bq) \stackrel{?}{<} (pb + aq)(bn) \\
                               &\cong qban + qb^2m \stackrel{?}{<} bnaq + b^2np \\
                               &\cong mq \stackrel{?}{<} np,
  \end{align*}
  where the last step involves subtraction of the same quantity ($qban$) from both sides, and also 
  division of both sides by $b^2$; note that neither of these operations affect the direction of any
  inequalities since the first operation is permissible in integer equations and $b^2 > 0
  \Rightarrow b^{-2} > 0$. \\

  But note that $y < z \Rightarrow mq < np,$ and we assumed $y < z$, so we have the first axoim. \\

  Now let's confirm that the second axoim holds, that is, $$x > 0, y > 0 \Rightarrow xy > 0.$$
  Suppose $x,y >0$ are as above. Then $xy = \frac{am}{nb}$. Note that since $x,y\in\Q > 0$, we have 
  that $a,b,m,n > 0$, so $\frac{am}{nb} > 0$.
\end{proof}
\begin{theorem}{Archimedean Property of $\Q$}
  $$x,y\in\Q, x > 0 \Rightarrow \exists n\in\Z^{>0} \st nx > y.$$
\end{theorem}
\begin{proof}
  Let $x,y$ be as in the theorem's statement. If $x \geq y$, then pick $n = 2$ and we're done, so
  suppose further that $x < y$. Now, since $x,y$ are rational, we have that $x = \frac{a}{b}, y =
  \frac{p}{q}$ for some $a,b,p,q \in\Z$ with $b,q\neq 0$. Let's characterize $n\st nx>y$, and then 
  show that it satisfies the conditions imposed in the theorem. Note that, from our definition of
  an ordering on $\Q$, we have that 
  \begin{align*}
    nx > y &\Rightarrow n\left(\frac{a}{b}\right) > \frac{p}{q} \\
           &\Rightarrow \frac{na}{b} > \frac{p}{q} \\
           &\Leftrightarrow naq > bp \\
           &\Rightarrow n > \frac{bp}{aq}.
  \end{align*}
  Since we have that $x\neq 0$, $a \neq 0$, and we already had that $b,q\neq 0$ by definition, so 
  we have that $$n' = \frac{bp}{aq}\in\Q.$$ Now, pick $n\in\Z^{>0}$ minimal such that $n > n'$, i.e.
  $n = \ceil{n'}$. \\

  Since $x > 0$, we never have to worry about some negative signs flipping the directions of a bunch
  of those inequalities.
\end{proof}

\begin{theorem}
  While $\R$ satisfies the \emph{least-upper bound} property, $\Q$ does not.
\end{theorem}
\begin{proof}
  We omit here the proof that $\R$ satisfies the least upper bound property. To prove the latter claim,
it suffices to find a subset of $\Q$ that does not have a least upper bound in $\Q$. Consider the set
$$E = \{x\in\Q \mid x^2 < 2 \}.$$ We have already shown this result in class, so I'll reproduce it 
here. \\

Suppose $\beta$ is an upper bound for $E$, and let $$\gamma = \frac{2\beta + 2}{\beta + 2}.$$ Then
$$\gamma^2 - 2 = 2\frac{\beta^2 - 2}{(\beta + 2)^2}.$$ Since $\beta^2 > 2$ since it is an upper bound,
we have that the whole expression is positive, and so we have that $\gamma^2 > 2 \Rightarrow \gamma$ is 
an upper bound. Now 
\begin{align*}
  \beta - \gamma &= \beta - 2\frac{\beta + 1}{\beta + 2} \\
                 &= \frac{\beta^2 + 2\beta - 2\beta - 2}{\beta + 2} \\
                 &= \frac{\beta^2 - 2}{\beta^2 + 2},
\end{align*}
so $\gamma < \beta$, which means that $\beta$ is not a least upper bound. Since $\beta$ is generic,
there can be no least upper bound.
\end{proof}

\begin{question}[2]
  A \emph{real polynomial} $p$ has the form $p(x) = p_nx^n + p_{n-1}x^{n-1} + \ldots + p_0$ for $p_0,
\ldots, p_n\in\R$ and some $n\in\Z^{\geq0}$. Unless $p$ is the zero polynomial we assume $p_n\neq 0$ 
and we say that $p$ has degree $n$. A real \emph{rational function} $p/q$ is the ratio of two polynomials
where $q$ is not identically zero. Let $F$ be the set of real rational functions. \\

\begin{qparts}
  \item Give $F$ the structure of a field; verify the field axioms.
  \item Define $p/q >0$ iff $$\frac{p}{q}(x) = \frac{p(x)}{q(x)} = \frac{p_nx^n + \ldots p_0}{q_mx^m
 + \ldots q_0}$$ satisfies $p_n/q_m > 0.$ Why is this well defined? Now, for $f,g\in F$ we say $f > g
\iff f- g > 0$. Prove that with this definition $F$ is an ordered field (Hint: You must prove that $F$
is an ordered set).
  \item Does $F$ satisfy the Archimedean property? Prove or disprove.
\end{qparts}
\end{question}
\begin{proof}
  \begin{qparts}
    \item
  Let $f/g, f'/g'\in F$. Then we define addition and multiplication as 
  \begin{align*}
    &\frac{f}{g} + \frac{f'}{g'} = \frac{fg' + gf'}{gg'},
    &\frac{f}{g}\frac{f'}{g'} = \frac{ff'}{gg'}.
  \end{align*}
  Then $0\in F$ is any $\frac{a}{b}\in F$ where $a$ is and $b$ is not identically zero, and $1\in F$
  is any $\frac{a}{b}\in F$ where $a$ is identically $b$. Let's quickly check these.
  \begin{align*}
    &\frac{0}{g} + \frac{f'}{g'} = \frac{0 + gf'}{gg'} = \frac{f'}{g'},
    &\frac{a}{a}\frac{f'}{g'} = \frac{af'}{ag'} = \frac{f'}{g'},
  \end{align*}
  as desired. Note also that inverses are also readily defined; for $\frac{f}{g}\in F$, we have that
  \begin{align*}
    &\frac{f}{g} + \frac{-f}{g} = \frac{fg - gf'}{gg} = \frac{0}{gg} = 0,
    &\frac{f}{g}\frac{g}{f} = \frac{fg}{gf} = \frac{1}{1},
  \end{align*}
  so now it remains to show that addition and multiplication are associative and commutative, and 
  that multiplication distributes over addition. Let $\frac{f''}{g''}\in F$. Then we want to show that
  $$\frac{f}{g} + \left(\frac{f'}{g'} + \frac{f''}{g''}\right) = \left(\frac{f}{g} + \frac{f'}{g'}\right)
  + \frac{f''}{g''}.$$ So
  \begin{align*}
    \frac{f}{g} + \left(\frac{f'}{g'} + \frac{f''}{g''}\right) &= \frac{f}{g} + \frac{f'g'' + g'f''}{
  g'g''} \\
  &= \frac{f(g'g'') + g(f'g'' + g'f'')}{gg'g''} \\
  &= \frac{g''(fg' + gf') + (gg')f''}{(gg')g''} \\
  &= \frac{fg' + gf'}{gg'} + \frac{f''}{g''} \\
  &= \left(\frac{f}{g} + \frac{f'}{g'}\right) + \frac{f''}{g''},
  \end{align*}
  as desired. We should also show that addition commutes, but this is obvious because $\R$ commutes and
  we just keep abusing properties of $\R$, and I'm also sick of typing a million fractions. For 
  multiplication, we want to show that $$\frac{f}{g} \times \left(\frac{f'}{g'} \times
  \frac{f''}{g''}\right) = \left(\frac{f}{g} \times \frac{f'}{g'}\right) \times \frac{f''}{g''}.$$ So
  \begin{align*}
    \frac{f}{g} \times \left(\frac{f'}{g'} \times \frac{f''}{g''}\right) 
  &= \frac{f}{g} \times \frac{f'f''}{ g'g''} \\
  &= \frac{ff'f''}{gg'g''} \\
  &= \frac{ff'}{gg'} \times \frac{f''}{ g''} \\
  &= \left(\frac{f}{g} \times \frac{f'}{g'}\right) \times \frac{f''}{g''},
  \end{align*}
  as desired. Now it just remains to show that addition distributes over multiplication. We'll just show
  left distributivity and claim that that is sufficient because everything commutes. We want to show 
  that $$\frac{f}{g}\times\left(\frac{f'}{g'}+\frac{f''}{g''}\right) = \left(\frac{f}{g}\times\frac{f'}
  {g'}\right) + \left(\frac{f}{g}\times\frac{f''}{g''}\right).$$ So
  \begin{align*}
    \frac{f}{g}\times\left(\frac{f'}{g'}+\frac{f''}{g''}\right) 
    &= \frac{f}{g}\times\frac{f'g'' + g'f''}{g'g''} \\
    &= \frac{f(g'g'') + g(f'g'' + g'f'')}{gg'g''} \\
    &= \frac{f'g + gf'}{gg'} + \frac{fg'' + gf''}{gg''} \\
    &= \left(\frac{f}{g}\times\frac{f'}{g'}\right) + \left(\frac{f}{g}\times\frac{f''}{g''}\right),
  \end{align*}
  as desired, which concludes our proof that $F$ is a field.

  \item This is well defined because division in $\R$ is well defined, so we won't have any pathological
    $p,q$ pairs appear (I don't really understand what's being asked here).\\

    We need to first show that exactly one of $(<,>,=)$ holds and that our order is \emph{transitive}.
    Note that since we've defined the order wholly in terms of operations on $\R$, we get the first 
    point for free. So it remains to show that $\frac{f}{g} < \frac{f'}{g'} \wedge \frac{f'}{g'} <
    \frac{f''}{g''} \Rightarrow \frac{f}{g} < \frac{f''}{g''}$. Suppose the antecedent.\\

    I see how to do this proof but I really hate type-setting fractions, so I'm going to leave it off.

  \item No, this isn't an Archimedean field, which should be apparent since the definition only deals 
    with the leading coefficients, leaving off a whole lot of important information. Consider 
    $$ 1- \frac{1}{x} = \frac{x - 1}{x} > 0,$$ since both leading coefficients are 1. This example 
    will prove that $F$ is Archimedean; we just have to show that $\frac{1}{x}, 1 > 0$ and that 
    $\forall n\Z^{>0}, 1 > n\frac{1}{x}$. The former point is obvious, since the leading coefficients
    in both $\frac{1}{x}$ and 1 are both positive. To show the second point, we just use our expansion
    from before, noting that $$1 - n\frac{1}{x} = \frac{x - n}{x} > 0,$$ again since the leading coefficients
    are both 1 and hence positive. So, we have produced a counterexample for the Archimedean property
    in $F$.
  \end{qparts}
\end{proof}

\begin{question}[3]
  Suppose $X,Y$ are sets. Define a function $X \to Y$ using set theory. Also, define function composition.
\end{question}
A \emph{function} is a set $f = \{(x,y)\} \subset X\times Y$ where each $x\in X$ is associated to at
most one $y\in Y$. Note that we use the standard definition of equality on tuples. We say that a
function is injective iff for $(x, y), (x', y')\in f$, we have that $y' = y \Rightarrow x' = x$, and
surjective iff $\forall y\in Y, \exists x\in X \st (x,y)\in f$. \\

\emph{Function composition} is an associative non-commutative operation $\circ: F\times F' \to F''$,
where $F$ is the set of functions from $X\to Y$, $F'$ is the set of functions from $Y\to Z$, and
$F''$ is the set of functions from $X\to Z$. 

\begin{question}[4]
  Consider the function 
  \begin{align*}
    f: (-1, 1) &\to \R \\
             x &\mapsto x^2.
  \end{align*}
  \begin{qparts}
    \item What are the domain, codomain, and range of $f$? 
    \item Is $f$ injective? surjective? bijective?
    \item What is $f([-1/3, 1/2])?$
    \item What is $f^{-1}(\{3,4,5\})$?
    \item How does the cardinality of $f^{-1}(y)$ depend on $y\in\R$?
    \item Is $f(1)$ defined?
  \end{qparts}
\end{question}
\begin{qparts}
  \item The domain of $f$ is $(-1, 1)$, the codomain of $f$ is $\R$, and the range, or image, of $f$
        is $[0, 1)$.
  \item $f$ is neither injective, surjective, nor bijective. $f$ is not injective, since $f(-0.5) =
        f(0.5)$, but $-0.5 \neq 0.5$. $f$ is not surjective, since it doesn't hit every point of $\R$;
        consider 19. Thus, $f$ is not bijective.
  \item $f([-1/3, 1/2]) = [1/6, 1/4]$, if overloading the definition of $f$ is permissible; to apply
        $f$ to an interval, we apply it to each point in the interval.
  \item $f^{-1}$ is not defined at these points, since none of these points are in the range.
  \item If $y$ is not in the range of $f$, then the fibre has size 0, since $f$ is undefined. If $y
        = 0$, then the fibre of $f$ has size one, and if $y \neq 0$ and $y$ is in the range, the fibre
        has size 2.
  \item $f(1)$ is not defined, since the domain of $f$ is the open interval $(-1, 1)$, which does not
        contain 1.
\end{qparts}

\end{document}
