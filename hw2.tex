\documentclass{assignment}
\usepackage{mymath}

\begin{document}
\header{Real Analysis I}{Homework 2}

\section*{Rudin}
%1,4,5,7,16,17
\begin{question}[1]
  If $r$ is rational, $r\neq 0$, and $x$ is irrational, prove that $r+x$ and $rx$ are irrational.
\end{question}
\begin{proof}
  Suppose not. That is, suppose that $r\neq 0\in\Q, x\notin\Q$, and $r+x$ and $rx\in\Q$; our aim is to
  show that this implies that $x\in\Q$, a contradiction. Since $r\in\Q$, we have that $\exists m,n
  \in\Z, n\neq 0 \st \frac{m}{n} = r$, and similarly, we have $\frac{a}{b} = r + x$ and $\frac{p}{q}
  = rx$. So then we have 
  \begin{align*}
    (r + x) - r &= \frac{a}{b} - \frac{m}{n} \\
                &= \frac{an - mb}{nb} \\
                &= x, 
  \end{align*}
  and so we've arrived at a contradiction for $r+x$, which means that $r+x\notin\Q$, i.e $r+x$ is irrational,
  since $an-mb \in \Z, np\neq 0\in\Z$ because $\Z$ is closed under addition and multiplication and 
  $x,y\in\Z \neq 0 \Rightarrow xy\neq 0$. Similarly, we have
  \begin{align*}
    rx / r &= \frac{p}{q} / \frac{m}{n} \\
           &= \frac{pn}{qm} \\
           &= x,
  \end{align*}
  and so we've arrived at a contradiction for $rx$, which means that $rx$ is irrational. This is because
  $pn\in\Z, qm\in\Z\neq 0$ since $r\neq 0 \Rightarrow m\neq 0$ and the definition of $q$, where $q\neq 0$.\\

  So we have the result.
\end{proof}

\begin{question}[4]
  Let $E$ be a non-empty subset of an ordered set; suppose that $\alpha$ is a lower bound of $E$ and
$\beta$ an upper bound. Prove that $\alpha\leq\beta$.
\end{question}
\begin{proof}
  %This proof relies on the transitivity of $\leq$.
  Let $\alpha, \beta, E$ be as in the question. Since $E$ is non-empty, let $x\in E$. Then by definition,
  $x\leq\beta$ and $\alpha\leq x$. Note that the definition of transitivity we rely on holds only for
  elements all in $E$, but this question isn't feasible without this natural step, since we have only 
  the axioms of an ordered set to use. So then by the transitivity of $\leq$, we have that 
  $$\alpha\leq x, x\leq\beta \Rightarrow \alpha\leq\beta.$$
\end{proof}

\begin{question}[5]
  Let $A$ be a nonempty set of real numbers which is bounded below. Let $-A$ be the set of all numbers
  $-x$, where $x\in A$. Prove that $$\inf A = -\sup(-A).$$
\end{question}
\begin{proof}
\end{proof}

\begin{question}[7]
  Fix $b > 1$, $y > 0$, and prove that there is a unique real $x$ such that $b^x = y$, by completing
  the following outline.
  \begin{qparts}
    \item For any positive integer $n$, $b^n - 1 \geq n(b-1)$.
    \item Hence $b - 1 \geq n(b^{1/n} - 1)$. 
    \item If $t > 1$ and $n > (b-1)/(t-1)$, then $b^{1/n} < t$.
    \item If $w$ is such that $b^w < y$, then $b^{w + (1/n)} > y$ for sufficiently large $n$; to see this,
      apply part (c) with $t = yb^{-w}$. 
    \item If $b^w > y$, then $b^{w - (1/n)} > y$ for sufficiently large $n$.
    \item Let $A$ be the set of all $w$ such that $b^w < y$, and show that $x = \sup A$ satisfies $b^x 
      = y$. 
    \item Prove that this $x$ is unique.
  \end{qparts}
\end{question}
\begin{proof}
  \begin{qparts}
  \item We conclude this by simply exponentiating the inequality in (a) by $\frac{1}{n}$, noting that
    since $\frac{1}{n} > 0$, the direction of the inequality is preserved.
  \item 
  \end{qparts}
\end{proof}

\begin{question}[16]
  Suppose $k\geq 3, x,y\in\R^k, |x - y| = d > 0$, and $r > 0$. Prove that 
  \begin{qparts}
    \item If $2r > d$, there are infinitely many $z\in\R^k$ such that $$|z-x| = |z - y| = r.$$
    \item If $2r = d$, there is exactly one such $z$.
    \item If $2r < d$, there is no such $z$.
  \end{qparts}
  How must these statements be modified if $k$ is 2 or 1?
\end{question}

\begin{question}[17]
  Prove that $$|x + y|^2 + |x - y|^2 = 2|x|^2 + 2|y|^2,$$ if $x,y\in\R^k$. Interpret this geometrically,
  as a statement about parallelograms.
\end{question}
\begin{proof}
  Let $x,y\in\R^k$. Then we have
  \begin{align*}
    |x+y|^2 + |x-y|^2 &= (x+y)^2 + (x-y)^2 \\
                      &= (x^2 + 2xy + y^2) + (x^2 - 2xy + y^2) \\
                      &= 2x^2 + 2y^2 \\
                      &= 2|x|^2 + 2|y|^2.
  \end{align*}
\end{proof}


\section*{Other Questions}
\begin{question}[1]
  Show that the rational numbers $\Q$ form an ordered field that satisfies the Archimedean property.
What property of $\R$ is not satisfied by $\Q$?
\end{question}

\begin{proof}
  TODO PROVE THIS
\end{proof}

\begin{theorem}
  While $\R$ satisfies the \emph{least-upper bound} property, $\Q$ does not.
\end{theorem}
\begin{proof}
  We omit here the proof that $\R$ satisfies the least upper bound property. To prove the latter claim,
it suffices to find a subset of $\Q$ that does not have a least upper bound in $\Q$. Consider the set
$$E = \{x\in\Q \mid x^2 < 2 \}.$$ We have already shown this result in class, so I'll reproduce it 
here. \\

Suppose $\beta$ is an upper bound for $E$, and let $$\gamma = \frac{2\beta + 2}{\beta + 2}.$$ Then
$$\gamma^2 - 2 = 2\frac{\beta^2 - 2}{(\beta + 2)^2}.$$ Since $\beta^2 > 2$ since it is an upper bound,
we have that the whole expression is positive, and so we have that $\gamma^2 > 2 \Rightarrow \gamma$ is 
an upper bound. Now 
\begin{align*}
  \beta - \gamma &= \beta - 2\frac{\beta + 1}{\beta + 2} \\
                 &= \frac{\beta^2 + 2\beta - 2\beta - 2}{\beta + 2} \\
                 &= \frac{\beta^2 - 2}{\beta^2 + 2},
\end{align*}
so $\gamma < \beta$, which means that $\beta$ is not a least upper bound. Since $\beta$ is generic,
there can be no least upper bound.
\end{proof}

\begin{question}[2]
  A \emph{real polynomial} $p$ has the form $p(x) = p_nx^n + p_{n-1}x^{n-1} + \ldots + p_0$ for $p_0,
\ldots, p_n\in\R$ and some $n\in\Z^{\geq0}$. Unless $p$ is the zero polynomial we assume $p_n\neq 0$ 
and we say that $p$ has degree $n$. A real \emph{rational function} $p/q$ is the ratio of two polynomials
where $q$ is not identically zero. Let $F$ be the set of real rational functions. \\

\begin{qparts}
  \item Give $F$ the structure of a field; verify the field axoims.
  \item Define $q/q >0$ iff $$\frac{q}{q}(x) = \frac{q(x)}{q(x)} = \frac{q_nx^n + \ldots q_0}{q_mx^m
 + \ldots q_0}$$ satisfies $p_n/q_m > 0.$ Why is this well defined? Now, for $f,g\in F$ we say $f > g
\iff f- g > 0$. Prove that with this definition $F$ is an ordered field (Hint: You must prove that $F$
is an ordered set).
  \item Does $F$ satisfy the Archimedean property? Prove or disprove.
\end{qparts}
\end{question}

\begin{question}[3]
  Suppose $X,Y$ are sets. Define a function $X \to Y$ using set theory. Also, define function composition.
\end{question}
A \emph{function} is a set $f = \{(x,y)\} \subset X\times Y$ where each $x\in X$ is associated to at most one 
$y\in Y$. We say that a function is injective iff for $(x, y), (x', y')\in f$, we have that $y' = y \Rightarrow
x' = x$, and surjective iff $\forall y\in Y, \exists x\in X \st (x,y)\in f$.

\begin{question}[4]
  Consider the function 
  \begin{align*}
    f: (-1, 1) &\to \R \\
             x &\mapsto x^2.
  \end{align*}
  \begin{qparts}
    \item What are the domain, codomain, and range of $f$? 
    \item Is $f$ injective? surjective? bijective?
    \item What is $f([-1/3, 1/2])?$
    \item What is $f^{-1}(\{3,4,5\})$?
    \item How does the cardinality of $f^{-1}(y)$ depend on $y\in\R$?
    \item Is $f(1)$ defined?
  \end{qparts}
\end{question}
\begin{qparts}
  \item $f(1)$ is not defined, since the domain of $f$ is the open interval $(-1, 1)$, which does not
        contain 1.
\end{qparts}

\end{document}
