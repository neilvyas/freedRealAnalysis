\documentclass{assignment}
\usepackage{mymath}
\usepackage{csquotes}

\begin{document}
\header{Real Analysis I}{Homework 4}

\section*{Rudin Problems}
\begin{question}[2.7*]
  Let $A_1, A_2, \ldots$ be subsets of a metric space.
\begin{qparts}
\item If $B_n = \bigcup_{i=1}^n A_i,$ prove that $\conj{B} = \bigcup_{i=1}^n \conj{A_i}.$
\item If $B_n = \bigcup_{i=1}^\infty A_i,$ prove that $\conj{B} \supset \bigcup_{i=1}^\infty \conj{A_i}.$
\end{qparts}
Show, by example, that this inclusion can be proper.
\end{question}
\begin{proof}\leavevmode
  \begin{qparts}
  \item $(\supset)$ Let $B_n$ be as above. Since for $i=1,\ldots, n$ we have that $A_i\subset B_n$,
    we have that $\conj{A_i}\subset\conj{B_n}$, and so $\bigcup_{i=1}^n\conj{A_i} \subset \conj{B_n}.$ \\

    $(\subset)$ Since $\conj{B_n} = B_n \cup B'_n$, it suffices to show that $B' \subset 
    \bigcup_{i=1}^n \conj{A_i}.$ Let $p\in B_n$ be a limit point of $B_n$; that is, $p\in B_n, B'_n$. 
    Let $N_m(p) = \{q\mid d(p,q) < 1/m\}$ be the $m$-neighborhood about $p$.
    Then $\forall m\in\N, \exists q_m \in N_m(p)$ such that $q_m\in B-n$, since $p$ is a limit point
    of $B_n$. By our definition of $N$, we have that $N_l \subset N_k$ for all $l\geq k$. So we have 
    that for $l \geq k$, if $q_k\notin A_i$, then $q_l\notin A_i$. \\

    In particular, for $i = 1,\ldots, n$ and some $k\in\N, l\geq k$, we have that $q_l\in A_i$, since
    we consider finitely many $A$ and countably many $q$ (i,e., this $k$ must exist). Thus, $p$ is a 
    limit point of $A_i$, and so $p\in\conj{A_i}$. So $\conj{B_n}\subset\bigcup_{i=1}^n A_i$.\\

    Thus, we have the claim.

  \item $\forall i, A_i \subset B = \bigcup_{j=1}^\infty A_j$. Thus $\conj{A_i} \subset \conj{B_n}$, 
    for all $i$. So then we have the claim.
    \end{qparts}
\end{proof}
Consider the following collection of sets $$B_n = \left(\frac{1}{n}, 1\right).$$ Then $(0,1) = \bigcup
_{i=1}^\infty B_n.$ But 
\begin{align*}
  \conj{(0,1)} &= [0,1] \\
               &\supset \bigcup_{i=1}^\infty \conj{(1/n, 1)} \\
               &= \bigcup_{i=1}^\infty [1/n, 1] \\
               &= (0,1],
\end{align*}
and so the inclusion is proper.

\begin{question}[2.9]
Let $E^0$ denote the set of all interior points of a set $E$. 
\begin{qparts}
\item Prove that $E^0$ is always open.
\item Prove that $E$ is open iff $E = E^0$.
\item If $G\subset E$ and $G$ is open, prove that $G\subset E^0$.
\item Prove that the complement of $E^0$ is the closure of the complement of $E$.
\item Do $E$ and $\conj{E}$ always have the same interiors?
\item Do $E$ and $E^0$ always have the same closures?
\end{qparts}
\end{question}
\begin{proof}\leavevmode
\begin{qparts}
  \item Suppose that $p\in\E^0$, and hence $p$ is an interior point of $E$. Then $\exists\delta > 0$ 
  such that $N_\delta(p) \subset E$, by definition, with $N$ open. Then $\forall x\in N_\delta(p),
  \exists \varepsilon > 0 \st N_\varepsilon(x)\subset N_\delta(p) \subset E$, and so $x$ is an interior
  point of $E$. Thus $N_\delta(p)\subset E^0$, and so $E^0$ is open.

  \item $(\Rightarrow)$ Suppose that $E$ is open. Then $p\in E\Rightarrow \exists\delta >0 \st N_\delta(p)
  \subset E$, and so $p\in E^0$. So then $E\subset E^0$. Conversely, $E^0\subset E$ obviously. So we 
  have that $E=E^0$. \\
  $(\Leftarrow)$ Suppose that $E^0 = E$. By part $(a)$ we have that $E^0$ is always open, so $E$ is 
  open.

  \item Let $p\in G$. Then, since $G$ is open, $\exists \delta > 0 \st N_\delta(p)\subset G\subset E$. 
  So, we have that $p$ is an interior point of $E$, and thus $p\in E^0$. So $G\subset E^0$.

  \item We want to show that $$(E^0)^c = \conj{E^c}.$$ $(\subset)$ Suppose that $p\in (E^0)^c$. Then,
  since $p$ is not an interior point of $E$, we have that $\forall N(p),  N\cap E^c \neq \emptyset.$
  So, either $p\in E^c$ or $\exists x\in N\cap E^c, x\neq p$, i.e $p\in E^c \cup (E^c)'$. Thus $p\in
  \conj{E^c}.$ \\

  $(\supset)$ You just perform the same argument, but with $p\in\conj{E^c}$ initially.

  \item No. Consider $E = (-1, 0)\cup(0,1)$. Then $E^0 = E$, but $\conj{E}^0 = (-1, 1)$, so they are
    different.

  \item No. Take $E = \Q$, then $\conj{E} = \R$ but $\conj{E^0}$ is empty.
  
  \end{qparts}
\end{proof}

\begin{question}[2.11]
For $x,y\in\R$, define $d(x,y)$ to be
\begin{qparts}
  \item $(x-y)^2,$ \\
  \item $\sqrt{|x - y|},$ \\
  \item $|x^2 - y^2|,$ \\
  \item $|x - 2y|,$ \\
  \item $\frac{|x-y|}{1 + |x - y|}.$ \\
\end{qparts}
Determine, for each of these, whether it is a metric or not.
\end{question}
\begin{proof}\leavevmode
  \begin{qparts}
  \item This isn't a metric, since it violates the triangle inequality. Consider $x=0, y=1, z=2$. 
    Then $d(x,y) + d(y,z) = 2$ but $d(x,z) = 4$, and $2 \not\geq 4$.

  \item Let $x=y$; obviously, $d(x,y) = 0$. Suppose $d(x,y) = 0$. Then $$\sqrt{|x-y|} = 0 \Rightarrow
    |x - y| = 0 \Rightarrow x -y = 0 \Rightarrow x = y,$$ as desired. We take the principal root, so
    $d$ is positive. $d$ is obviously symmetric, since $|\cdot, \cdot|$ is symmetric. Thus, it only
    remains to check the triangle inequality.\\

    Let $x,y,z\in\R$. Then note that $d(x,y) + d(y,z) \geq d(x,z) \iff (d(x,y) + d(y,z))^2 \geq d(x,z)^2,$ 
    since $d$ is positive. Further, since $d$ is positive, note that $(d(x,y) + d(y,z))^2 \geq 
    d(x,y)^2 + d(y,z)^2$. Thus, it suffices to prove the latter. But the latter is just the statement
    $$|x - y| + |y - z| \geq |x - z|, $$ which is true since $|\cdot, \cdot|$ is itself a distance 
    and thus obeys the triangle inequality.

    \item Consider $x = 1, y=-1$. Then $x\neq y$ but $d(x,y) = 0$, and so we have that this $d$ is
not a metric.

    \item This is obviously not a metric since it isn't symmetric, i.e. $\exists x,y \st d(x,y) \neq 
d(y,x)$. The counterexample we give is $x = 1, y = 2$. Then $d(x,y) = 3$ but $d(y,x) = 0$;
additionally, we see that this $d$ violates positivity as well.

    \item From an earlier homework, we had that $$d'(x,y) = \frac{d(x,y)}{1 + d(x,y)}$$ is a metric
iff $d(x, y)$ is a metric, and we have that $d(x,y) = |x-y|$ is a metric since it is the canonical
metric on $\R$, so we have that this is a metric.  
  \end{qparts}
\end{proof}

\begin{question}[2.13]
  Construct a compact set of real numbers whose limit points form a countable set.
\end{question}
\begin{proof}
  Consider the set $\{0\}$. 
\end{proof}
\begin{proof}
  We have that a set in $\R$ is compact iff it is bounded and closed. \\

  Consider the set $$E = \left\{\frac{1}{n} + \frac{1}{m}\ \middle|\ n,m\in\N \right\}.$$ Then the limit
  points are $E' = \left\{\frac{1}{n}\mid n\in\N\right\},$ which is obviously countable. This set is
  closed, since it contains all its limit points, and bounded, obviously, so it is compact. 
\end{proof}

\begin{question}[2.14]
  Give an example of an open cover of the segment $(0,1)$ which has no finite subcover.
\end{question}
\begin{proof}
  Let's just use the obvious one. Consider the collection of sets given by
  $$B_n = \left(\frac{1}{n}, 1\right).$$ Then $\bigcup_{n\in\N}B_n$ is a cover of $(0,1)$, but for any
  finite $m$, the maximum index number in a finite subcollection, we have that $\frac{1}{m}\in(0,1)$,
  but not in the union of the finite subcollection. So this open cover has no finite subcover.
\end{proof}

\begin{question}[2.15]
  Show that Theorem 2.36 and its Corollary become false (in $\R$, for example) if the word
``compact'' is replaced by ``closed'' or ``bounded.''
\end{question}
\begin{displayquote}
  \textbf{Theorem 2.36} \\
  If $\{K_a\}$ is a collection of compact subsets of a metric space $X$ such that the intersection of
every finite subcollection of $\{K_a\}$ is nonempty, then $\bigcap K_a$ is nonempty. \\

  \textbf{Corollary}\\
If $\{K_n\}$ is a sequence of nonempty compact sets such that $K_n \supset K_{n+1}$, then $\bigcap_1
^\infty K_n$ is not empty.  
\end{displayquote}
The following proofs have a nice dichotomy of ``running away'' in the first case and ``collapsing to
nothing'' in the second.
\begin{proof}
  Replace the word ``compact'' by ``closed'' in the above theorem. Then consider the sequence of closed
  sets given by $$C_n = [n, \infty), n\in\N.$$ Note that each $C_n$ is closed since its complement is 
  open. So the intersection of every finite subcollection is nonempty, since $$\bigcap_i^j C_i = 
  [N, \infty),$$ where $N = \max\{n_i,\ldots,n_j\}$. But since there is no largest integer, 
    $$\bigcap_{n\in\N}C_n = \emptyset.$$
\end{proof}
\begin{proof}
  Replace the word ``compact'' by ``bounded'' in the above theorem. Then consider the sequence of 
  bounded sets given by $$B_n = [1 - \frac{1}{n}, 1), n\in\N.$$ Then note that as above, the intersection
  of every finite subcollection is nonempty, but the arbitrary intersection is empty (this is proved
  via a variant of the standard $\delta-\varepsilon$ mumbo-jumbo that gets so much mileage in analysis,
  but I'll omit it here).
\end{proof}

\begin{question}[2.16]
  Regard $\Q$ as a metric space, with $d(p,q) = |p - q|$. Let $$ E = \{p\in\Q\mid 2 < p^2 < 3\}.$$
Show that $E$ is closed and bounded in $\Q$, but that $E$ is not compact. Is $E$ open in $\Q$?
\end{question}
\begin{proof}
  We have from prior results that $\sqrt{2}$ and $\sqrt{3}$ are $\notin\Q$. Obviously, $E$ is bounded,
  since it sits inside the interval $(-42, 42)\cap\Q$. Let's first answer the questions about openness
  and then the question about compactness. \\

  We want to show that $E$ is open in $\Q$. Write 
  $$E = \Q \cap [(-\sqrt{3}, -\sqrt{2}) \cup (\sqrt{3}, \sqrt{2})].$$ Then since $E$ is an intersection
  of an open set in $\R$ with $\Q$, $E$ must be open in $\Q$. Now, since $\Q$ is dense in $\R$, the 
  closure of $E$ in $\R$ is $[-\sqrt{3}, -\sqrt{2}] \cup [\sqrt{3}, \sqrt{2}]$. But the closure of 
  $E$ in $\Q$ is $\Q \cap [(-\sqrt{3}, -\sqrt{2}) \cup (\sqrt{3}, \sqrt{2})]$, since we have that\
  $\sqrt{2}, \sqrt{3} \notin \Q$. So since $E = \conj{E}$ in $\Q$, we have that $E$ is closed in $\Q$. \\

  Let's prove the statement about compactness now. Take the the cover of $E$ given as follows:
  $$B_n = \Q \cap [(-\sqrt{3} + \frac{1}{n}, -\sqrt{2} - \frac{1}{n}) \cup (\sqrt{2} + \frac{1}{n}, \sqrt{3} - \frac{1}{n})].$$
  Note that $$E\subset\bigcup_{n\in\N}B_n,$$ but $E$ is not a subset of a union of any finite 
  subcollection of $\{B_n\}$, say $\bigcup_i^j B_i$. Then let $N = \max\{n_i,\ldots,n_j\}$. So then
  $\Q\cap (\sqrt{3} - \frac{1}{n}, \sqrt{3}) \not\subset \bigcup_i^j B_i$, and so $E\not\subset\bigcup_i^j B_i$.
\end{proof}

\section*{Other Problems}
\begin{question}[1]
  \begin{qparts}
    \item Let $(X_1, d_1)$ and $(X_2, d_2)$ be metric spaces. Construct a metric on the Cartesian
product $X_1 \times X_2$.
    \item Let $X$ be the set of triangles in the Euclidean plane $E^2$. Construct a metric on $X$
which captures the intuitive notion of when two triangles are close.  
  \end{qparts}
\end{question}
\begin{proof}\leavevmode
  \begin{qparts}
  \item Let $$d\left((x_1, y_1), (x_2,y_2)\right) = d_1(x_1, x_2) + d_2(y_1, y_2),$$ for $(x,y)\in X_1\times X_2$.
    Now it just remains to check the three criteria for $d$ in order to determine whether or not it is 
    a distance. It satisfies positivity since two positive numbers add to 0 iff both are 0, and both
    $d_1, d_2$ satisfy positivity since they are distances. It is also obviously symmetric since both
    $d_1, d_2$ are symmetric, again since they are both distances. \\

    Now we must check the triangle inequality. Let $(x_1, y_1), (x_2, y_2), (x_3, y_3) \in X_1\times 
    X_2$. Then 
    \begin{align*}
      d\left((x_1, y_1), (x_2, y_2)\right) + d\left((x_2, y_2), (x_3, y_3)\right) &= 
      \left[d_1(x_1, x_2) + d_2(y_1, y_2)\right] + \left[d_1(x_2, x_3) + d_2(y_2, y_3)\right] \\
      &= \left[d_1(x_1,x_2) + d_1(x_2, x_3)\right] + \left[d_2(y_1,y_2) + d_2(y_2, y_3)\right] \\
      &\geq d_1(x_1, x_3) + d_2(y_1, y_3) \\
      &= d((x_1, y_1), (x_3, y_3)),
    \end{align*}
    as desired. So, finally, we conclude that $d$ as defined is a metric.

   \item We can construct the ``obvious'' but limited metric, which is to take the midpoint of each 
     triangle (averaging the vertices) and then computing the usual distance in $\E^2$ on the resulting
     pair of vectors. But if we want to truly capture the intuitive notion of close, we should make
     some effort to associate the corresponding points together. \\

     Let $\Sigma$ be the set of all permutations of $\{1,2,3\}$. Then for any two triangles $T_1 = 
     (x_1, x_2, x_3)$ and $T_2 = (y_1, y_2, y_3)$, we have that 
     $$d_T(T_1, T_2) = \min_{\sigma\in\Sigma}\sum_{i=1}^3d(x_i, y_{\sigma(i)}).$$

     I don't really want to check that this is a metric, mainly because of verifying the triangle
     inequality bit, but since it's in terms of the canonical $d$ it should be a metric.
  \end{qparts}
\end{proof}

\begin{question}[3]
  For each of the following statements write its negation and, for parts (a) and (b), state whether
the negation is true or false. You need not provide a proof.
\begin{qparts}
  \item For every metric space $(X, d)$ there exist $p,q\in X$ such that $d(p,q) > 0.$
  \item There exists a metric space $(X,d)$ such that every subset of $X$ is closed.
  \item Let $f: [0,1]\to\R$. For all $x_0\in[0,1]$ and all $\varepsilon\in\R^{>0}$, there exists
$\delta\in\R^{>0}$ such that if $|x - x_0| < \delta$ then $|f(x) - f(x_0)| < \varepsilon.$
\end{qparts}
\end{question}
\begin{proof}\leavevmode
  \begin{qparts}
  \item The negation of this statement is $$\exists (X,d) \text{ a metric space } \st \forall p,q\in
    X, d(p,q) \leq 0.$$ We can amend the final statement to be $d(p,q) = 0$, since $d$ is positive in 
    a metric space. \\
    
    Take the metric space $X = \{x\}$ equipped with the discrete metric (note that, since $X$ has
    only one element, this metric is uniquely determined). So note that there do not exist $p,q\in X$
    such that $d(p,q) > 0$, since $\forall p,q \in X, d(p,q) = 0$ since $p = q$. Thus, we have that the 
    negation is true.
    
  \item The negation of this statement is $$\forall \text{ metric spaces }(X,d),\ \exists G\subset X
    \st G \text{ is open.}$$ Taking the same example $(X,d)$ used above, we have that the negation is false.

  \item The negation of this statement is $$\exists x\in[0,1], \varepsilon\in\R^{>0} \st \forall \delta
    \in\R^{>0}, |x - x_0| < \delta \wedge |f(x) - f(x_0)| \geq \varepsilon.$$ The original statement 
    is just a statement of continuity, so the truth value depends on the function $f$. For example,
    $f(x) = x^2$ is continuous, so the original statement holds, but it's easy to construct a pathological
    function that just takes the values $0$ and $\infty$ in order to falsify the original statement.
  \end{qparts}
\end{proof}

\begin{question}[3]
  What is wrong with the following argument which ``proves'' that for any positive integer $n$, if
there is a room of people with at least one redhead, then everyone in the room is a redhead?
\begin{displayquote}
  The statement is clearly true for $n = 1$. Now given the truth of the statement for $n$, suppose a
room has $n + 1$ people and includes a redhead. Suppose that not everyone is blessed with red hair.
Then putting aside one of the non-redheads, we have a room with $n$ people in it, including a
redhead.  By the induction hypothesis, all $n$ people are redheads. Now bring back the non-redhead
who stepped aside and have some other person step aside. Again we have $n$ people in the room with
all but one redhead. So by induction all $n$ are redheaded. Since the person who stepped aside is
also redheaded, we have proved that all $n + 1$ people are redheaded. \\
\end{displayquote} 
\end{question}
\begin{proof}
  We will use the mechanism of the proof to induce a contradiction. Suppose we have three people in a
  room, $a,b,c$, and $c$ is a redhead. Suppose the IH for $n=2$. Then have $a$ exit. Since only $b,c$
  remain, then by the IH everyone in the room must be a redhead. In particular, $b$ must be a redhead.
  But we assumed $b$ not to be a redhead, so we have arrived at a contradiction.
\end{proof}
\end{document}
