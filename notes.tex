\documentclass{notes}

\title{Real Analysis 1}
\subtitle{Dr. Freed}
\author{Neil Vyas}
\class{M 365C}
\date{Spring 2016}

\begin{document}

\maketitle
\tableofcontents

%TODO: definitions are
\begin{aside}{Theorems, Lemmas, etc.}
A Theorem is a proven statement in mathematics. \\
A Lemma is a smaller result on the way to proving a Theorem. \\
A Proposition is a smaller result that isn't deserving of the title ``Theorem.'' \\
A Corollary is a result that follows easily from another Theorem.
\end{aside}

\section{Number Systems}
\subsection{The Integers} % Lecture 1 - Jan 19th, 2016
Natural Numbers $\N$, $\Z^{>0} = \{1, 2, 3, \ldots\}$ equipped with the operation of addition.
Addition is a binary operation that is \emph{commutative} and \emph{associative.} Note that the 
identity element is zero, which is not in $\N$. \\

Non-negative integers $\Z^{\geq 0} = \{0 , 1, 2, \ldots\}$ is equipped with the same operation as 
$\N$, but also has the identity element 0. \\

The integers $\Z = \{\ldots, -1, 0, 1, \ldots\}$ are a countably infinite set that include inverse
elements for each integer. \\

\begin{aside}{Cardinalities}
Is $\Z$ bigger than $\Z^{\geq 0}$? How can we answer this question? We will approach it by examining 
some functions between the sets. We say that these sets are equal in ``size'' if we have a \emph{bijection}
between them; remember that a function is a bijection if it is both \emph{injective} and \emph{surjective}\\

The canonical inclusion is injective but not surjective, but we can still easily construct a bijection.
Consider that we really can find two equivalence classes of ``equal'' size in each set: in $\Z$, we have
the equivalence relation of sign, and in $\N$, we have the equivalence relation of even-ness. So send
the even numbers to half them, and the odd numbers to the negative numbers. \\
\end{aside}

Note that the integers equipped with addition and multiplication are just missing multiplicative inverses;
addition and multiplication are both associative, commutative, and have an identity element. We can
introduce a bigger number system to contain these multiplicative inverses, which we will call the rationals:
$$\Q = \{\frac{a}{b} \mid a, b \in \Z, b \neq 0\}.$$ 

How can we represent or describe $\Q$, starting from $\Z$? It looks like we can just do $\Z\times\Z^{\times}$, 
but this won't be correct since many points will be ``the same'' rational number. \\

%TODO
%put the defns of commutativity, associativity, identity, inverses here
%put the defns of rings and fields here

So $\Q$ is a field, and $\Z$ is a ring. \\

But $\Q$ still isn't big enough. What does this mean? For example, consider algebraic equations: relations
between elements of the field given by the operations of the field. We say $\Q$ isn't ``big enough'' 
because there are certain algebraic equations that do not admit rational solutions. \\

%TODO: format this
Let's consider the equation $x^2 = a$, for some $a\in\Q$. If $a = 4$, then we're golden, but what if 
$a = 2$? \\

\begin{theorem}
  There does not exist $x\in\Q \st x^2 = 2$. \\
\end{theorem}

We're going to prove the following lemmas on the way to proving this result:

\begin{lemma}
  $x\in\Q \Rightarrow \exists a,b\in\Z \text{ not even } \st x = \frac{a}{b}.$
\end{lemma}

\begin{proof}
  Since $x\in\Q$, we can choose $a,b\in\Z$ with $b > 0$ and minimal. Then if $a,b$ are both even, $b$ is
  not minimal, since we can halve both $a$ and $b$, but this is a contradiction. \\
\end{proof}

\begin{lemma}
  $a\in\Z, a^2 \text{ even} \Rightarrow a \text{ even}.$
\end{lemma}

\begin{proof}
The contrapositive is $a\text{ odd}\Rightarrow a^2\text{ odd},$ so let's try and prove that. Suppose
$a\in\Z$ and $a$ is odd. Then $\exists k\in\Z \st a = 2k + 1$. So 
\begin{align*}
  a^2 &= (2k + 1)^2 \\
  &= 4k^2 + 4k + 1 \\
  &= 2(2k^2 + 2k) + 1,
\end{align*}
and thus $a^2$ is odd. 
\end{proof}

So let's prove the theorem now. 

\begin{proof}
  Suppose not. That is, suppose TODO. Write $x = \frac{a}{b}, a,b\in\Z,
  b >0$, with not both $a,b$ even. Since $x^2 = 2$, we have $a^2 = 2b^2$, so by the lemma, since $a^2$ 
  is even, $a$ is even. But then $b^2$ must be even, since we can write $a = 2a^\prime$, and the result
  follows from there. So then both $a$ and $b$ are even, which is a contradiction. \\
\end{proof}

\section{Sequences}
We say a sequence is a function from $\N\to\Q$. Consider $$a_n = 1 + \frac{1}{1!} + \frac{1}{2!} + \ldots$$ 
Note that this is a function of the number of terms (in $\N$), and each term is in $\Q$. But this sequence
is ``converging'' to $e$, which is not in $\Q$. Additionally, consider the following sequence:
$$b_n = (1 + \frac{1}{n})^n.$$
Note that both these sequences converge to $e$, but the second converges much faster.\\

We will define the real numbers $\R$ in a similar manner, with these limit processes.

%lecture 2
\section{Relations}
\subsection{Orderings}
%defn
Let $S$ be a set. Then a \emph{relation} $R$ is a subset of $S\times S$. An \emph{order} is a relation
$R$ with the following properties: (for $x,y\in\R$)
\begin{enumerate}
\item $x,y\in S,$ exactly one of $x = y, x < y, x> y$ holds. Note that equality is in $S\times S$, not
  equality in the sense of our relation.
\item $x,y,z\in S, x<y, y<z \Rightarrow x<z$ \emph{(transitivity)}
\end{enumerate}

A set $S$ with an order is called an \emph{ordered set}. For example, $\Z\times\Z$ equipped with the
lexicographical ordering is an ordered set. However, we can see that $$(a,b) < (a', b') \Leftrightarrow
a + b < a' + b'$$ is not an ordering on $\Z\times\Z$ by considering the points $(3,2)$ and $(2,3)$. 

\subsection{Ordered Fields}
%defn ordered field
Let $F$ be a field equipped with an order $R$. Then $F$ is an \emph{ordered field} iff for $x,y,z\in F$, 
we have 
\begin{enumerate}
    \item $y < z \Rightarrow x+ y < x + z$
    \item $x > 0, y>0 \Rightarrow xy > 0$.
\end{enumerate}
Examples of ordered fields include $\Q$ with the usual ordering, Note that $\C$ is not an ordered field,
since $i^2 = -1$, which violates the second condition.

\begin{proposition}
  Let $F$ be an ordered field. Then 
  \begin{enumerate}
    \item $x > 0 \Rightarrow -x < 0$
    \item $x > 0, y<z \Rightarrow xy < xz$
    \item $x,y < 0 \Rightarrow xy > 0$
    \item $x < 0, y < z \Rightarrow xy > xz$
    \item $x\neq 0 \Rightarrow x^2 > 0$
    \item $1 > 0$
    \item $0 < x < y \Rightarrow 0 < \frac{1}{y} < \frac{1}{x}$
    \item $x,y < 0 \Rightarrow xy > 0$
  \end{enumerate}
\end{proposition}

\begin{proof}
  Get this from the book!
\end{proof}

\subsection{Complete Fields}
Note that we showed that $\Q$ is ``incomplete,'' since we had algebraic equations that had no solutions
in $\Q$ as well as sequences that did not converge in $\Q$. Now, we'll try to show this same statement 
using just the concept of order. \\

%defn
Let $S$ be an ordered set. 
\begin{enumerate}
  \item If $E\subset S, \beta\in S,$ and $\forall x\in E, x\leq\beta,$ then $\beta$ is an \emph{upper bound}
    of $E$. If such a $\beta$ exists, we say that $E$ is \emph{bounded above}.
  \item If $\alpha\in S$ is an upper bound of $E\subset S$ and $\forall \gamma < \alpha, \gamma$ is not
    an upper bound of $E$, we say that $\alpha$ is the \emph{least upper bound} of $E$.
  \item If $\forall E\subset S$ bounded above there exists a least upper bound, we say that $S$ has the
    \emph{least upper bound property}.
\end{enumerate}
The least upper bound is called the supremum, denoted $\sup$, while the greatest lower bound is called the
infimum, denoted $\inf$. \\

Consider $$E = \{x\in\Q \mid x^2 < 2 \} \subset S = \Q.$$ Note that this set is obviously bounded, above
and below. But $\sup E$ and $\inf E$ are not attained, since $\sqrt{2}\notin\Q$. Let's make this more
rigorous. \\

\begin{proposition}
  Write $\Q = E\cup E'$, where $E$ is as above and $E' = \{x\in\Q \mid x^2 > 2\}$. Then $E$ is bounded
  above and below but neither $\sup E$ nor $\inf E$ are attained, and $E'$ is unbounded.
\end{proposition}

\begin{proof}
  Suppose $\beta$ is an upper bound for $E$, and let $$\gamma = \frac{2\beta + 2}{\beta + 2}.$$ Then
  $$\gamma^2 - 2 = 2\frac{\beta^2 - 2}{(\beta + 2)^2}.$$ Since $\beta^2 > 2$ since it is an upper bound,
  we have that the whole expression is positive, and so we have that $\gamma^2 > 2 \Rightarrow \gamma$ is 
  an upper bound. Now 
  \begin{align*}
    \beta - \gamma &= \beta - 2\frac{\beta + 1}{\beta + 2} \\
                   &= \frac{\beta^2 + 2\beta - 2\beta - 2}{\beta + 2} \\
                   &= \frac{\beta^2 - 2}{\beta^2 + 2},
  \end{align*}
  so $\gamma < \beta$, which means that $\beta$ is not a least upper bound. Since $\beta$ is generic,
  there can be no least upper bound.
\end{proof}

\begin{theorem}
  There exists an ordered field $\R$ with the least upper bound property. Furthermore, $\Q\subset\R$
  as a subfield, i.e. there is an injective function $I:\Q\to\R$ which is compatible with the axoims
  of an ordered field, and $\R$ is unique.
\end{theorem}

%lecture 3
\begin{theorem}[Archimedean Property for $\R$]
  $x,y\in\R, x > 0 \Rightarrow \exists n\in\Z^{>0} \st nx > y$.
\end{theorem}
\begin{proof}
  Let $E = \{nx\mid n\in\Z^{>0}\}$. Suppose not. That is, suppose that $\forall n\in\Z^{>0}, y > nx$. 
  Thus, $y$ is an upper bound of $E$. By the least upper bound property of $\R$, let $\alpha = \sup E$.
  Then $\alpha - x < \alpha$ is not an upper bound for $E$. Thus, $\exists n\in\Z^{>0}\st nx > \alpha
  - x$. So we have that $$(n+1)x = nx + x > \alpha,$$ using the distributive property and the fact that
  $x > 0$, so we can add it to both sides and preserve the direction of the inequality. But since 
  $(n+1)\in\Z^{>0}$, we have that $\alpha$ is not an upper bound. 
\end{proof}
So we say that $\R$ is an \emph{Archimedean Field}.

\begin{theorem}[$\Q$ Dense in $\R$]
  $x,y\in\R, x < y \Rightarrow \exists r\in\Q \st x < r < y$.
\end{theorem}
This should be obvious, since we can just generate a rational from an irrational by truncation. So,
we just need to truncate far enough into the expansion to ensure that the truncation lies in the 
interval.Anyway, let's make this rigorous.
\begin{proof}
  Let $x,y$ be as above. Then we seek $m,n\in\Z, n\neq 0 \st x < \frac{m}{n} < y$. In order to get 
  more precision in the number of decimal places, we have to find $n$ large enough. So, we need that
  $$\frac{1}{n} < y-x,$$ i.e. choose $n$ sufficiently large that this condition holds. Note that we 
  can find such an $n$ because of the Archimedean property; now it only remains to find an appropriate
  $m$. We need $m$ to satisfy $nx < m$ and $m < ny$, both of which are separately satisfiable by the
  Archimedean property. \\

  Let's try and satisfy both separately and show that we can ``merge'' them into a single $m$. Let's 
  start with $m_2 < ny$. Since the Archimedean property lets us produce ``bigger'' numbers, let's 
  negate this inequality, yielding $-m_2 > -ny$. Now, by the Archimedean property, we have $m_1, 
  -m_2 > 0$ that satisfy both conditions and $m2 < 0 < m1$. Let $$M = \{m\in\Z\mid m_2 \leq m \geq m_1\}$$.
  Note that this set is non-empty and finite. However, we can't characterize the $m$ to pick from this
  set in terms of $m_1, m_2$ because we have no real constraints on $m_1$ and $m_2$, so they can be
  arbitrarily large or small. \\

  So we'll just specify an algorithm to find $m\in M$. Choose $m\in E \st m > nx \wedge (m - 1 \leq 
  nx \vee m = m_2)$.  Note that we can find this $m$ by starting at $\frac{m_2}{n}$ and incrementing
  by $\frac{1}{n}$. \\

  Does this work? We need to show that $x < \frac{m}{n} < y$; let's start with the $ < y$ direction.
  Given our conditions for $m$, we have that 
  \begin{align*}
    m - 1 &\leq nx \Rightarrow \\
    m &\leq nx + 1 \Rightarrow \\
    \frac{m}{n} &\leq x + \frac{1}{n} \\
                &< x + (y - x) \\
                &= y.
  \end{align*}
  We get the other direction for free from the conditions on $m$.
\end{proof}

\section{Spaces}
$\R^n$ is a \emph{vector space}, while $\E^n$ is a \emph{Euclidean space}. A Euclidean space is a 
vector space with the additional structure of an \emph{inner product}, $\ip{\cdot}{\cdot}:
\R^n\times\R^n\to\R$. The standard inner product is $\ip{x}{y} = \sum_{i=1}^n x_iy_i$. Any 
inner product must satisfy \emph{positivity}, \emph{reflexivity}, and \emph{symmetry}. Inner products
are \emph{linear}. \\

With this notion of inner product, we can go on to define some other useful properties of vectors,
like the length of a vector and the angle between two vectors. We can also derive a canonical \emph{norm}
as the length, with $||x||^2 = \left<x,x\right>$, which satisfies the \emph{triangle inequality}. \\

We also have that $$\cos\theta = \frac{\ip{\xi}{\eta}}{||\xi||\ ||\eta||}.$$

We say that two vectors $\xi, \eta$ are \emph{orthogonal} $\iff \ip{\xi}{\eta} = 0$. \\

So an inner product gives us a notion of \emph{geometry}: we have distance, angle, and length. \\

%lecture 3
There's some nonsense distinction between points and vectors, but I have never
really seen this before, and it seems like its only use is semantic disambiguation, so I'm tempted
to not include it with the tag ``I don't care.'' %\\

\begin{defn}
  A \emph{distance}, or a \emph{metric} $d: V \times V \to \R^{\geq 0}$ satisfies, for $p,q,r,\in V$
  and $V$ a set, usually an inner product space,
  \begin{enumerate}
    \item $d(p,q) \geq 0$, or \emph{positivity};
    \item $d(p,q) = 0 \iff p = q$, the \emph{identity of indiscernibles};
    \item $d(p,q) = d(q,p)$, or \emph{symmetry};
    \item $d(p,q) \leq d(p,r) + d(r,q)$, or the \emph{triangle inequality}.
  \end{enumerate}
\end{defn}

\begin{defn}
  A \emph{metric space} is a set $V$ together with a metric $d: V \times V \to \R^{\geq 0}$.
\end{defn}

\begin{proposition}
  A standard distance function for $p,q\in \E^n$ is $$d(p,q) = || p - q ||.$$
\end{proposition}
\begin{proof}
  Mostly work from definitions.
  \begin{enumerate}
      \item blah
      \item boooooring
      \item bleh
      \item We want to show that $$||\eta + \xi|| \leq |\eta|| + ||\eta||,$$ for $$r = p + \xi, q = 
        r + \eta.$$ So
        \begin{align*}
          ||\eta + \xi||^2 &= \ip{\eta + \xi}{\eta + \xi} \\
                           &= ||\xi||^2 + 2\ip{\xi}{\eta} + ||\eta||^2 \note{(bi-linearity of
                                $\ip{\cdot}{\cdot}$)} \\
                           &\leq ||\xi||^2 + 2||\xi||||\eta|| + ||\eta||^2 \note{(Cauchy-Schwartz)} \\
                           &= \left(||\eta|| + ||\xi||\right)^2,
        \end{align*}
        and since norms are positive, we have that $$||\eta + \xi|| \leq ||\eta|| + ||\xi||,$$ as
        desired.
  \end{enumerate}
\end{proof}

\begin{theorem}{Cauchy-Schwartz Inequality}
  For any vectors $\xi, \eta$, we have that $$|\ip{\xi}{\eta}| \leq ||\eta|| ||\xi||.$$
\end{theorem}
\begin{proof}
  \emph{(NB: The proof shown in class kinda sucks and doesn't generalize past 2 dimensions, so I'll
  omit it here. There's a nice slick proof that I'll steal from my linear algebra notes or
  something.)} \\

  Since norms are non-negative, we will again work by squaring both sides, which is an equaivalent
  statement. So we have to show $$|\ip{\xi}{\eta}|^2 \leq ||\eta||^2 ||\xi||^2.$$

  Hint for a nicer proof: consider the projection of $\eta$ onto the subspace spanned by $\xi +
  t\eta$.
\end{proof}

\begin{example}{Metric Spaces}
  \begin{enumerate}
      \item The empty set with the trivial metric.
      \item Any set $X$ equipped with 
        $$d_c(p,q) = \begin{cases} 0 &\text{ if } p=q \\ c\in\R^{>0} &\text{ if } p\neq q
        \end{cases},$$ the discrete map.
      \item The set of lines in $\E^n$ equipped with the smallest angle distance.  
      \item The set of lines in $\E^n$ equipped with the unit rectangle distance (define this).
  \end{enumerate}
\end{example}

%lecture 5
\begin{proposition}
  Let $X$ be the set of sequences $a_1, a_2, \ldots$, where $a_i \in \{0,1\}$. Then $X$ is uncountable.
\end{proposition}
\begin{proof}
  By Cantor's diagonal argument.
\end{proof}

\begin{defn}
  Let $(X,d)$ be a metric space, $E\subset X$ a subsapce, $p\in X$, $r\in\R^{>0}$. Then 
  \begin{enumerate}
    \item The \emph{open ball} of radius $r$ about $p$ is $$B_r(p) = \{ q\in X\mid d(p,q) < r.$$
    \item The \emph{deleted ball} of radius $r$ about $p$ is $$B_r'(p) = B_r(p)\setminus\{p\}.$$
    \item $E$ is \emph{open} if $$\forall p\in E, \exists \varepsilon > 0 \st B_\varepsilon(p) \subset E.$$
    \item $E$ is \emph{closed} iff $E^c$ is open.
    \item $E$ is \emph{bounded} if $\exists p\in X, R>0 \st E\subset B_R(p)$.
    \item $E$ is \emph{dense} iff $\forall p\in X, \varepsilon > 0, B_r(p)\cap E$ is non-empty.
    \item A point $p\in X$ is 
      \begin{itemize}
        \item an \emph{interior point} of $E$ iff $\exists \varepsilon > 0 \st B_\varepsilon(p) \subset E$.
        \item a \emph{limit point} of $E$ iff $\forall r > 0, B_r'(p)\cap E$ is non-empty.
        \item an \emph{isolated point} of $E$ if $\exists \varepsilon > 0 \st B_\varepsilon(p)\cap E = \{p\}$.
      \end{itemize}
  \end{enumerate}
\end{defn}

\end{document}
