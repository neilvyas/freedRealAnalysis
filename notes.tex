\documentclass{notes}

\title{Real Analysis 1}
\subtitle{Dr. Freed}
\author{Neil Vyas}
\class{M 365C}

\begin{document}

\maketitle
\tableofcontents

%TODO: definitions are
A Theorem is a proven statement in mathematics.
A Lemma is a smaller result on the way to proving a Theorem.
A Proposition is a smaller result that isn't deserving of the title ``Theorem.''
A Corollary is a result that follows easily from another Theorem.

\section{Number Systems}
\subsection{The Integers} % Lecture 1 - Jan 19th, 2016
Natural Numbers $\N$, $\Z^{>0} = \{1, 2, 3, \ldots\}$ equipped with the operation of addition.
Addition is a binary operation that is \emph{commutative} and \emph{associative.} Note that the 
identity element is zero, which is not in $\N$. \\

Non-negative integers $\Z^{\geq 0} = \{0 , 1, 2, \ldots\}$ is equipped with the same operation as 
$\N$, but also has the identity element 0. \\

The integers $\Z = \{\ldots, -1, 0, 1, \ldots\}$ are a countably infinite set that include inverse
elements for each integer. \\

\begin{aside}{Cardinalities}
Is $\Z$ bigger than $\Z^{\geq 0}$? How can we answer this question? We will approach it by examining 
some functions between the sets. We say that these sets are equal in ``size'' if we have a \emph{bijection}
between them; remember that a function is a bijection if it is both \emph{injective} and \emph{surjective}\\

The canonical inclusion is injective but not surjective, but we can still easily construct a bijection.
Consider that we really can find two equivalence classes of ``equal'' size in each set: in $\Z$, we have
the equivalence relation of sign, and in $\N$, we have the equivalence relation of even-ness. So send
the even numbers to half them, and the odd numbers to the negative numbers. \\
\end{aside}

Note that the integers equipped with addition and multiplication are just missing multiplicative inverses;
addition and multiplication are both associative, commutative, and have an identity element. We can
introduce a bigger number system to contain these multiplicative inverses, which we will call the rationals:
$$\Q = \{\frac{a}{b} \mid a, b \in \Z, b \neq 0\}.$$ 

How can we represent or describe $\Q$, starting from $\Z$? It looks like we can just do $\Z\times\Z^{\times}$, 
but this won't be correct since many points will be ``the same'' rational number. \\

%TODO
%put the defns of commutativity, associativity, identity, inverses here
%put the defns of rings and fields here

So $\Q$ is a field, and $\Z$ is a ring. \\

But $\Q$ still isn't big enough. What does this mean? For example, consider algebraic equations: relations
between elements of the field given by the operations of the field. We say $\Q$ isn't ``big enough'' 
because there are certain algebraic equations that do not admit rational solutions. \\

%TODO: format this
Let's consider the equation $x^2 = a$, for some $a\in\Q$. If $a = 4$, then we're golden, but what if 
$a = 2$? \\

\begin{theorem}
  There does not exist $x\in\Q \st x^2 = 2$. \\
\end{theorem}

We're going to prove the following lemmas on the way to proving this result:

\begin{lemma}
  $x\in\Q \Rightarrow \exists a,b\in\Z \text{ not even } \st x = \frac{a}{b}.$
\end{lemma}

\begin{proof}
  Since $x\in\Q$, we can choose $a,b\in\Z$ with $b > 0$ and minimal. Then if $a,b$ are both even, $b$ is
  not minimal, since we can halve both $a$ and $b$, but this is a contradiction. \\
\end{proof}

\begin{lemma}
  $a\in\Z, a^2 \text{ even} \Rightarrow a \text{ even}.$
\end{lemma}

\begin{proof}
The contrapositive is $a\text{ odd}\Rightarrow a^2\text{ odd},$ so let's try and prove that. Suppose
$a\in\Z$ and $a$ is odd. Then $\exists k\in\Z \st a = 2k + 1$. So 
\begin{align*}
  a^2 &= (2k + 1)^2 \\
  &= 4k^2 + 4k + 1 \\
  &= 2(2k^2 + 2k) + 1,
\end{align*}
and thus $a^2$ is odd. 
\end{proof}

So let's prove the theorem now. 

\begin{proof}
  Suppose not. That is, suppose TODO. Write $x = \frac{a}{b}, a,b\in\Z,
  b >0$, with not both $a,b$ even. Since $x^2 = 2$, we have $a^2 = 2b^2$, so by the lemma, since $a^2$ 
  is even, $a$ is even. But then $b^2$ must be even, since we can write $a = 2a^\prime$, and the result
  follows from there. So then both $a$ and $b$ are even, which is a contradiction. \\
\end{proof}

\section{Sequences}
We say a sequence is a function from $\N\to\Q$. Consider $$a_n = 1 + \frac{1}{1!} + \frac{1}{2!} + \ldots$$ 
Note that this is a function of the number of terms (in $\N$), and each term is in $\Q$. But this sequence
is ``converging'' to $e$, which is not in $\Q$. Additionally, consider the following sequence:
$$b_n = (1 + \frac{1}{n})^n.$$
Note that both these sequences converge to $e$, but the second converges much faster.\\

We will define the real numbers $\R$ in a similar manner, with these limit processes.

\end{document}
