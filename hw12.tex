\documentclass{assignment}
\usepackage{mymath}

\no{12}
\class{Real Analysis I}

\begin{document}
\maketitle
%ch 6: 1,2,4,7,8
\section*{Rudin Problems}
\begin{question}[6.1]
 Suppose $\alpha$ increases on $[a,b], a \leq x_0 \leq b, \alpha$ is continuous at $x_0, f(x_0) =
1$, and $f(x) = 0 $ if $x \neq x_0$. Prove that $f\in\mathcal{R}{\alpha}$ and that $\int f\ d\alpha
= 0$. 
\end{question}

\begin{question}[6.2]
 Suppose $f \geq 0$, $f$ is continuous on $[a,b]$, and $\int_a^b f(x)\ dx = 0$. Prove that $f(x) =
0$ for all $x\in[a,b]$. 
\end{question}

\begin{question}[6.4]
 If $f(x) = 0$ for all $x\in\R\setminus\Q$, $f(x) = 1$ for all $x\in\Q$, prove that $f \notin
\mathcal{R}$ on $[a,b]$ for any $a < b$.
\end{question}

\begin{question}[6.7]
 Suppose $f$ is a real function on $(0,1]$ and $f\in\mathcal{R}$ on $[c, 1]$ for every $c > 0$.
Define $$\int_0^1 f(x)\ dx = \lim_{c\to 0}\int_c^1 f(x)\ dx,$$ if this limit exists and is finite.
\begin{qparts}
  \item If $f\in\mathcal{R}$ on $[0,1]$, show that this definition of the integral agrees with the
old one.
  \item Construct a function $f$ such that the above limit exists, although it fails to exist with
$|f|$ in place of $f$.
\end{qparts} 
\end{question}

\begin{question}[6.8]
  
\end{question}

\section*{Other Problems}
\begin{question}[1]
  For any nonnegative integer $n$ prove from the definition that 
$$\int_0^1 x^n\ dx$$ exists. Can you also compute the integral without using the fundamental theorem
of calculus?
\end{question}
\begin{theorem}
  Let $f: [a,b] \to \R$ be a (bounded) function. Then 
  \begin{enumerate}
    \item If $f$ is \emph{continuous}, it is Riemann integrable. 
    \item If $f$ is \emph{monotone}, it is Riemann integrable.
  \end{enumerate}
\end{theorem}
\begin{proof}
  Proved in class.
\end{proof}
\begin{proof}
  Note that existence is trivial. Fix $n \in\N$. Take $f: [0,1] \to \R$, $f(x) = x^n$. Since $f$ is
a polynomial, it is continuous on the interval, and so by the theorem we have that it is Riemann
integrable.
\end{proof}

\begin{question}[2]
  Prove or disprove each statement. You may have to use some statements you know from calculus.
\begin{qparts}
  \item If $f:[0,1] \to \R$ is Riemann integrable, then $f$ is continuous.
  \item IF $f:[a,b]\to\R$ is differentiable at all interior points of $[a,b]$, then $\int_a^b f\ dx$
exists.
  \item For $n = 0, 1, 2, \dots$ define $f_n: [0,1]\to \R$ by the formula $f_n(x) = \frac{x^n}{n!}$.
Then $$\int_0^1 \sum_{n=0}^\infty f_n(x)\ dx = \sum_{n=0}^\infty \int_0^1  f_n(x)\ dx.$$
\end{qparts}
\end{question}
\begin{theorem}[Criterion for Riemann Integrability]
  $f$ is Riemann integrable iff 
  $$\forall \varepsilon > 0, \exists P\in \mathcal{P} \st U(P,f) - L(P,f) < \varepsilon.$$
\end{theorem}
\begin{proof}
  Proved in class.
\end{proof}
\begin{proof}\leavevmode
  \begin{qparts}
   \item False. Let $f$ be identically 0 on the interval except that $f(1/2) = 1$. Obviously, $f$
is not continuous. Now we want to find a partition $P$ such that the theorem holds. Note that the
lower sum $L(P, f) = 0$ for all $P$. Fix $\varepsilon > 0$. Then we need to ensure that $U(P,f) <
\varepsilon$. But this is easy. Take $0 < \delta < \varepsilon$, and then $P = \{1/2(1  - \delta),
1/2(1  + \delta) \}$. So then $U(P,f) = \delta < \varepsilon,$ as desired. So $f$ is Riemann
integrable but not continuous.
   \item If $f$ is differentiable on $(a,b)$, then it is continuous at all points in $(a,b)$.
With some judicious handwaving involving compactness of $[0,1]$, we have that $f$ is continuous on
$[a,b]$, and so by the first theorem from class, we have that $f$ is Riemann integrable.
  \end{qparts}
\end{proof}

\begin{question}[3]
  Consider $$f(x) = \begin{cases} 1/q, &x = p/q. p, q \geq 0 \text{ relatively prime;} \\
                                  0, &x \in \R \setminus \Q. \end{cases}$$
  Compute the lower and upper integrals on $[0,1]$. Is $f$ integrable? 
\end{question}
\end{document}
