\documentclass{notes}
\usepackage{mymath}

\renewcommand{\emph}[1]{\textbf{#1}}

\begin{document}
\begin{center}
  \textsf{\textbf{\Huge Exam 1 Review}}
\end{center}
\section{Sets and Fields}
%cardinality
\begin{defn}
  An infinite set $X$ is \emph{countable} if we can give a bijection $\varphi: \N \to X$, and \emph{uncountable}
if we can give an injection $\psi: \R \to X$.
\end{defn}

%algebraic definition
\begin{defn}
  A \emph{field} is a set $X$ together with two operations $+, \cdot: X\times X \to X$ that satisfy 
the following properties:
\begin{enumerate}
  \item Associativity of\/ $(+)$
  \item Existence of Additive Identity
  \item Existence of Additive Inverses
  \item Commutativity of\/ $(\cdot)$
  \item Associativity of\/ $(\cdot)$
  \item Existence of Multiplicative Identity
  \item Existence of Multiplicative Inverses
  \item Distributive Law: $$x \cdot (y + z) = (x \cdot y) + (x \cdot z)$$ 
\end{enumerate}
\end{defn}

\begin{defn}
  An \emph{ordering} is a relation $R$ with the following properties:
\begin{enumerate}
  \item Exactly one of $x = y, x < y, x > y$ holds;
  \item Transitivity: $x < y, y < z \Rightarrow x < z$.
\end{enumerate}
\end{defn}

\begin{defn}
  An \emph{ordered field} is a field together with an ordering that is compatible with the structure
of the field; that is, it satisfies 
\begin{enumerate}
  \item $y < z \Rightarrow x + y < x + z$
  \item $x > 0, y > 0 \Rightarrow xy > 0$.  
\end{enumerate}
\end{defn}

\subsection{$\R$ and $\Q$}
\begin{defn}
  The \emph{Archimedean property} for an ordered field $X$ is $$\forall x, y \in \R, x > 0 
\Rightarrow \exists n \in \N \st nx > y.$$
\end{defn}
\begin{theorem}
  Both $\Q$ and $\R$ satisfy the Archimedean property.
\end{theorem}

\begin{defn}
  An \emph{upper bound} $\beta \in X$ for a set $E \subset X$ satisfies $\forall e \in E, e \leq \beta$.
The definition for a \emph{lower bound} is analogous.
\end{defn}
\begin{defn}
  The \emph{least upper bound} $\gamma$ of a set $E$, denoted $\gamma = \sup E$, satisfies $\forall
\alpha < \gamma, \alpha \text{ is not an upper bound of } E.$ The definition for \emph{greatest 
lower bound}, denoted $\inf E$, is analogous. A set has the least upper bound property if it has a
least upper bound.
\end{defn}
\begin{theorem}
  $\R$ satisfies the least upper bound property.
\end{theorem}

\begin{defn}
  A set $X$ is \emph{dense} in another set $Y$ if $$\forall x,y \in Y, x < y, \exists r \in X \st 
x < r < y.$$
\end{defn}
\begin{theorem}
  $\Q$ is dense in $\R$.
\end{theorem}

\section{Metric spaces}
\begin{defn}
  A \emph{metric space} is a set $X$ equipped with a distance function $d: X\times X \to \R$, denoted
$(X, d)$.
\end{defn}

\begin{defn}
  A \emph{distance function} is a function $d: X \times X \to \R$ that satisfies
\begin{enumerate}
  \item \emph{positivity}: $d(p,q) \geq 0$.
  \item \emph{symmetry}: $d(p,q) = d(q,p)$.
  \item \emph{identity of indiscernibles}: $d(p,q) = 0 \iff p = q$
  \item \emph{the triangle inequality}: $d(p,q) \leq d(p,r) + d(r, q)$.
\end{enumerate}
\end{defn}
\subsection{Sets in Metric Spaces}
%open closed bounded compact
\begin{defn}
  A point $p$ is an \emph{interior point} of $X$ iff $\exists \varepsilon > 0 \st B_\varepsilon(p)
\subset X$.
\end{defn}
\begin{defn}
  A point $p$ is a \emph{limit point} of a set $X$ iff every $\varepsilon > 0$ ball contains a 
point of $X$.
\end{defn}
\begin{defn}
  A set is \emph{open} iff every point is an interior point.
\end{defn}
Note that closed and open are not mutually exclusive. In particular, in any topological space $X$, both
the empty set $\emptyset$ and the whole space $X$ are closed and open.
%closure
\begin{defn}
  The \emph{closure} of a set $A$, denoted $\conj{A}$, is the union of the set with its limit points;
mathematically, $\conj{A} = A \cup A'$.
\end{defn}
\begin{proposition}
  The closure of a set is always closed.
\end{proposition}

\subsection{Sequences}
\begin{defn}
  A \emph{sequence} in $X$ is a function $p: \N \to X$, often written as $\{p_n\}$, where $p_n = p(n)$. 
\end{defn}

\begin{defn}
  A sequence $\{p_n\} \subset X$ is a \emph{Cauchy sequence} iff $$\forall \varepsilon > 0, \exists 
N\in\N \st \forall m,n \geq N, d(p_m, p_n) < \varepsilon.$$
\end{defn}

\begin{defn}
  A sequence $\{p_n\} \subset X$ \emph{converges} to $p\in X$, denoted $p_n \to p$, iff 
$$\forall \varepsilon > 0, \exists N\in\N \st \forall n \geq N, d(p_n, p) < \varepsilon.$$
\end{defn}

\end{document}
