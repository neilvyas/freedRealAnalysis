\documentclass{assignment}
\usepackage{mymath}

\no{5}
\class{Real Analysis I}

\begin{document}
\maketitle

\section*{Rudin Problems}
%Chapter 2 (page 44): 19*
%Chapter 3 (page 78): 1, 2, 3, 20*, 23*, 24* (extra credit)
\begin{question}[2.19*]
  \begin{qparts}
   \item If $A$ and $B$ are disjoint closed sets, prove that they are separated.
   \item Prove the same for disjoint open sets.
   \item Fix $p\in X, \delta > 0.$ Then let $$A = \{q\in X \mid d(p,q) < \delta\};$$ let $B$ be the 
   same but with $<$ replaced with $>$. Prove that $A$ and $B$ are separated.
   \item Prove that every connected metric space with at least two points is uncountable. \emph{Hint:
   use part (c).}
  \end{qparts} 
\end{question}
\begin{proof}\leavevmode
  \begin{qparts}
    \item Let $A, B$ be disjoint closed sets. Since $A, B$ are closed, we have that $\conj{A} = A$, and 
     similarly for $B$. Since $A, B$ are disjoint, we have that their intersection is empty. Putting this
     together, we have that $$\conj{A} \cap B =  A \cap B = \emptyset,\ A \cap \conj{B} = A \cap B = \emptyset,$$
     and thus by the definition of separated, $A$ and $B$ are separated.

    \item Let $A,B$ be disjoint open sets. Suppose not; that is, suppose that $A,B$ are not separated.
    This means that at least one of $\conj{A} \cap B$ or $A \cap \conj{B}$ is non-empty; WLOG suppose
    that $\conj{A} \cap B \neq \emptyset$. Then let $x\in \conj{A} \cap B$. Since $B$ is open and 
    $x\in B$, there is a neighborhood $N(x) \subset B$. Since $x\in\conj{A}$, we have that $N(x) \cap
    A \neq \emptyset$. So then we have that $$A \cap N(x) \subset A \cap B \Rightarrow A \cap B \neq 
    \emptyset,$$ since $N(x)\subset B$. But this is a contradiction. So, $A,B$ must be separated.

    \item We will use the result from $(b)$, so we must show that $A,B$ are disjoint and open. Obviously,
    $A$ is open since it is an open ball. To show that $B$ is open, note that its complement is 
    $\conj{A}$, which is always closed, so then $B$ is open. Now we need to show that $A,B$ are disjoint.
    Suppose not; that is, suppopse that $A\cap B \neq \emptyset$. Then let $x\in A\cap B$. Since $x\in A$
    and $x\in B$, we have that $$d(p, x) < \delta,\ d(p,x) > \delta,$$ a contradiction by the definition
    of the order on $\R$. So the intersection must be empty. Thus, $A,B$ are disjoint open sets, and
    so by the result from $(b)$ they are separated.

    \item Let $X$ be a connected metric space with at least two points. Let $x_1, x_2 \in X$. In order
    to use the result from $(c)$, we need to construct two disjoint open sets; the obvious choice
    look like open balls about $x_1, x_2$. In order to show that $X$ is uncountable, we must provide
    an injection from an uncountable set into $X$. \\

    Let $t\in (0, 1/2)$. This is the uncountable set we will construct an injection into $X$ with.
    So to each $t$ associate $r_t = t d(x_1, x_2),$ and let $$A_t = \{q\in X\mid d(x_1, q) < r_t \},
    \ B = \{q\in X\mid d(x_2, q) < r_t\}.$$ Obviously, $A,B$ are disjoint by the triangle inequality, 
    and open by definition. So, they are separated. \\

    Either $X = A_t \cup B_t$, or $\exists x_t \in X \st x_t \notin A_t \cup B_t$. Since $X$ is connected,
    it cannot be the union of separated sets, so the latter must be true. We can pick a specific
    class of $x_t$ by further specifying that $d(x_1, x_t) = r_t$. Then to each $t$ we can associate
    a unique $x_t\in X$, so we have the injection we desired, and thus $X$ is uncountable.  
  \end{qparts} 
\end{proof}

\newpage
\begin{question}[3.1]
  Prove that the convergence of $\{s_n\}$ implies the convergence of $\{|s_n|\}$. Is the converse
  true?
\end{question}
\begin{proof}
  Suppose that $\{s_n\}$ converges, that is, $\{s_n\} \to s$ for some $s$. Then by definition, given
  $\varepsilon > 0, \exists N\in\N \st \forall n \geq N, |s_n - s| < \varepsilon$. But since 
  $$||x| - |y|| \leq |x - y|,$$ we have that $||s_n| - |s|| \leq |s_n - s| \leq \varepsilon$, and so
  $|s_n| \to |s|$.
\end{proof}
\begin{proof}
  No, the converse is not true, i.e. we can have conditionally convergent sequences. Consider the
alternating sequence $s_n = (-1)^n$. Then $s_n$ never converges, but $|s_n|$ is identically 1.
\end{proof}

\begin{question}[3.2]
  Calculate $\lim_{n\to\infty}(\sqrt{n^2 + n} - n)$.
\end{question}
\begin{proof}
  \begin{align*}
    \sqrt{n^n + n} - n &= \frac{\sqrt{n^2 + n} - n}{\sqrt{n^2 + n} + n}\sqrt{n^2 + n} + n \\
                       &= \frac{n^2 + n - n^2}{\sqrt{n^2 + n} + n} \\
                       &= \frac{1}{\sqrt{1 + \frac{1}{n}} + 1},
  \end{align*}
  and $$\lim_{n\to\infty} \frac{1}{\sqrt{1 + \frac{1}{n}} + 1} = \frac{1}{2}.$$
\end{proof}

\begin{question}[3.3]
  If $s_1 = \sqrt{2}$, and $$s_{n+1} = \sqrt{2 + \sqrt{s_n}},\ n\in\N,$$ prove that $\{s_n\}$ converges,
  and that $s_n < 2$ for $n\in\N$.
\end{question}
\begin{proof}
 This just turns into gross algebra and I would rather play some violin, so I'll do that instead. Sorry! 
\end{proof}

\begin{question}[3.20*]
  Suppose $\{p_n\}$ is a Cauchy sequence in a metric space $X$, and some subsequence $\{p_{n_l}\}$ 
  converges to a point $p\in X$. Prove that the full sequence $\{p_n\}$ converges to $p$.
\end{question}
\begin{proof}
  This is actually fairly straightforward; since the sequence is Cauchy, we can control the distance 
  of any term in the sequence from any term in the subsequence, and thus by triangle inequality control 
  the distance to $p$. So, let's do it. \\

  We want to find $$N\in\N \st \forall n \geq N, d(p_n, p) < \varepsilon,$$ for any $\varepsilon > 0$.
  Fix $\varepsilon > 0.$ Since $\{p_{n_l}\} \to p$, we have that $$\exists N_l \in \N \st \forall n 
  \geq N_l, d(p_n \in \{p_{n_l}\}, p) < \frac{1}{2}\varepsilon.$$ Since $\{p_n\}$ is Cauchy, 
  $$\exists N_C \in \N \st \forall n,m \geq N_C, d(p_n, p_m) < \frac{1}{2}\varepsilon.$$ So then 
  \begin{align*}
    d(p_n, p) &\leq d(p_n, p_m) + d(p_m, p) \\
              &< \varepsilon,
  \end{align*}
  for $p_n \in \{p_n\}, p_m \in \{p_{n_l}\}, N = \max(N_l, N_C)$.
\end{proof}

\begin{question}[3.23*]
  Suppose $\{p_n\}$ and $\{q_n\}$ are Cauchy sequences in a metric space $X$. Show that the sequence 
  $\{d(p_n, q_n)\}$ converges. \emph{Hint:} For any $m,n$, $$d(p_n, q_n) \leq d(p_n, p_m) + d(p_m, q_m)
    + d(q_m, q_n);$$ it follows that $$|d(p_n, q_n) - d(p_m, q_m)|$$ is small if $m$ and $n$ are large.
\end{question}
\begin{proof}
  While we don't have that the two sequences necessarily converge, since $d: X\times X \to \R$, it 
  suffices to show that the sequence of differences is Cauchy, and then since $\R$ is complete, by 
  another result we have that the sequence of differences is convergent. Fix $\varepsilon > 0$.
  Then we want to show that 
  $$\exists N\in\N \st \forall n,m \geq N, |d(p_n, q_n) - d(p_m, q_m)| < \varepsilon.$$

  Since $\{p_n\}, \{q_m\}$ are Cauchy, we have that 
  $$\exists N_P \in \N \st \forall n,m \geq N_P, d(p_n, p_m) < \frac{1}{2}\varepsilon,$$ 
  and anologously for $q$, with $N_Q$ in place of $N_P$. \\

  By the hint, and since $d$ is positive, we have that 
  \begin{align*}
    |d(p_n, q_n) - d(p_m, q_m)| &\leq |d(p_n, p_m) + d(q_m, q_n)| \\
                                &< \varepsilon,
  \end{align*}
  for $n,m > \max(N_p, N_q)$, as desired.
\end{proof}

\begin{question}[3.24*]
  Let $X$ be a metric space.
  \begin{qparts}
  \item Call two Cauchy sequences $\{p_n\}, \{q_n\}$ in $X$ \emph{equivalent} iff $$\lim_{n\to\infty}
    d(p_n, q_n) = 0.$$ Prove that this is an equivalence relation.
  \item Let $X^*$ be the set of all equivalence classes so obtained. If $P, Q\in X^*, \{p_n\}\in P,
    \{q_n\}\in Q$, define $$\Delta(P, Q) = \lim_{n\to\infty} d(p_n, q_n);$$ by Exercise 23, this limit
    exists. Show that the number $\Delta(P,Q)$ is unchanged if $\{p_n\}$ and $\{q_n\}$ are replaced
    by equivalent sequences, and hence that $\Delta$ is a distance function in $X^*$.
  \item Prove that the resulting metric space $X^*$ is complete.
  \item For each $p\in X$, there is a Cauchy sequence all of whose terms are $p$; let $P_p$ be the element
    of $X^*$ which contains this sequence. Prove that $$\Delta(P_p, P_q) = d(p, q)$$ for all $p,q\in X$.
    In other words, the mapping $\phi$ defined by $\phi(p) = P_p$ is an isometry of $X$ into $X^*$.
  \item Prove that $\phi(X)$ is dense in $X^*$, and that $\phi(X) = X^*$ if $X$ is complete. By $(d)$,
    we may identify $X$ and $\phi(X)$ and thus regard $X$ as embedded in the complete metric space $X^*$.
    We call $X^*$ the \emph{completion} of $X$.
\end{qparts}
\end{question}

\newpage
\section*{Other Problems}
\begin{question}[1]
Evaluate the following limits (in R) or show that the limit does not exist.
\begin{qparts}
  \item $\lim_{n\to\infty}(-1)^n$
  \item $\lim_{n\to\infty} \frac{n^2 - n}{n^3 + n}$
  \item $\lim_{n\to\infty} n^{1/n}$
\end{qparts}
\end{question}
\begin{proof}\leavevmode
  \begin{qparts}
  \item For any $n\in\N$, $d((-1)^n, (-1)^{n+1}) = 1$, so this limit does not exist by oscillation. 
    That is, we cannot satisfy the definition for any $\varepsilon < 1$.
  \item $$\frac{n^2 - n}{n^3 + n} = \frac{1/n - 1/n^2}{1 + 1/n^2},$$ and 
    $$\lim_{n\to\infty} \frac{1/n - 1/n^2}{1 + 1/n^2} = 0.$$
  \item This sequence should converge to 1, so noting that $n^{(1/n)} > 1$ for all $n$, we WTS that 
    for any $\varepsilon > 0$, we have $N$ such that $$(1 + \varepsilon)^N \geq N.$$ This winds up being
    kind of ugly to work out, as you might expect, so I will omit giving a form for $N$ here, but it
    should be apparent that such an $N$ exists, since the LHS is exponential and the RHS is linear. \\

    Alternatively, if we have L'hopital's rule, we can take the natural log of the limit and then evaluate,
    yielding 1. But since we don't have that machinery yet, we can't use that as a proof.  
  \end{qparts}
\end{proof}

\begin{question}[2]
  Let $(X, d)$ be a metric space, $\{p_n\} \subset X$ a convergent sequence with $p_n \to p$, and 
  $\{q_n\} \subset X$ a convergent sequence with $q_n \to q$. Prove that $d(p_n, q_n) \to d(p, q)$
  (This last convergence takes place in $\R$).
\end{question}
\begin{proof}
  We want to find $N_t$ such that $$\forall i \geq N_t,i\ |d(p_i, q_i) - d(p, q)| < \varepsilon,$$ for
  any $\varepsilon > 0$.
  Since $\{p_n\}\to p, \{q_n\}\to q$, there are $N, M \in \N$ such that $$\forall n \geq N,
  m \geq M,\ d(p_n, p) < \frac{1}{2}\varepsilon,\ d(q_m, q) < \frac{1}{2}\varepsilon.$$ So then
  \begin{align*}
    d(p_n, q_m) &\leq d(p_n, p) + d(p, q_m) \\
                &\leq d(p_n, p) + d(p,q) + d(q, q_m) \\
                &< \varepsilon + d(p,q),
  \end{align*}
  and thus 
  \begin{align*}
    |d(p_n, q_m) - d(p,q)| &< |\varepsilon + d(p,q) - d(p,q)| \\
                           &= \varepsilon,
  \end{align*}
  as desired. Note that we can resolve any indexing issues $(p_n, q_m)$ by simply indexing both sequences
  by $N_t = \max(n,m)$.
\end{proof}

\end{document}
