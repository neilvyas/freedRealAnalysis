\documentclass{assignment}
\usepackage{mymath}

\no{5}
\class{Real Analysis I}

\begin{document}
\maketitle

\section*{Rudin Problems}
%Chapter 2 (page 44): 19*
%Chapter 3 (page 78): 1, 2, 3, 20*, 23*, 24* (extra credit)
\begin{question}[2.19]
  \begin{qparts}
   \item If $A$ and $B$ are disjoint closed sets, prove that they are separated.
   \item Prove the same for disjoint open sets.
   \item Fix $p\in X, \delta > 0.$ Then let $$A = \{q\in X \mid d(p,q) < \delta\};$$ let $B$ be the 
   same but with $<$ replaced with $>$. Prove that $A$ and $B$ are separated.
   \item Prove that every connected metric space with at least two points is uncountable. \emph{Hint:
   use part (c).}
  \end{qparts} 
\end{question}

\begin{question}[3.1]
  Prove that the convergence of $\{s_n\}$ implies the convergence of $\{|s_n|\}$. Is the converse
  true?
\end{question}
No, the converse is not true, i.e. we can have conditionally convergent sequences.
\begin{proof}
  
\end{proof}

\begin{question}[3.2]
  Calculate $\lim_{n\to\infty}(\sqrt{n^2 + n} - n)$.
\end{question}

\begin{question}[3.3]
  If $s_1 = \sqrt{2}$, and $$s_{n+1} = \sqrt{2 + \sqrt{s_n}},\ n\in\N,$$ prove that $\{s_n\}$ converges,
  and that $s_n < 2$ for $n\in\N$.
\end{question}

\begin{question}[3.20]
  Suppose $\{p_n\}$ is a Cauchy sequence in a metric spae $X$, and some subsequence $\{p_{n_l}\}$ 
  converges to a point $p\in X$. Prove that the full sequence $\{p_n\}$ converges to $p$.
\end{question}

\begin{question}[3.23]
  Suppose $\{p_n\}$ and $\{q_n\}$ are Cauchy sequences in a metric space $X$. Show that the sequence 
  $\{d(p_n, q_n)\}$ converges. \emph{Hint:} For any $m,n$, $$d(p_n, q_n) \leq d(p_n, p_m) + d(p_m, q_m)
    + d(q_m, q_n);$$ it follows that $$|d(p_n, q_n) - d(p_m, q_m)|$$ is small if $m$ and $n$ are large.
\end{question}

\begin{question}[3.24]
  Let $X$ be a metric space.
  \begin{qparts}
  \item Call two Cauchy sequences $\{p_n\}, \{q_n\}$ in $X$ \emph{equivalent} iff $$\lim_{n\to\infty}
    d(p_n, q_n) = 0.$$ Prove that this is an equivalence relation.
  \item Let $X^*$ be the set of all equivalence classes so obtained. If $P, Q\in X^*, \{p_n\}\in P,
    \{q_n\}\in Q$, define $$\Delta(P, Q) = \lim_{n\to\infty} d(p_n, q_n);$$ by Exercise 23, this limit
    exists. Show that the number $\Delta(P,Q)$ is unchanged if $\{p_n\}$ and $\{q_n\}$ are replaced
    by equivalent sequences, and hence that $\Delta$ is a distance function in $X^*$.
  \item Prove that the resulting metric space $X^*$ is complete.
  \item For each $p\in X$, there is a Cauchy sequence all of whose terms are $p$; let $P_p$ be the element
    of $X^*$ which contains this sequence. Prove that $$\Delta(P_p, P_q) = d(p, q)$$ for all $p,q\in X$.
    In other words, the mapping $\phi$ defined by $\phi(p) = P_p$ is an isometry of $X$ into $X^*$.
  \item Prove that $\phi(X)$ is dense in $X^*$, and that $\phi(X) = X^*$ if $X$ is complete. By $(d)$,
    we may identify $X$ and $\phi(X)$ and thus regard $X$ as embedded in the complete metric space $X^*$.
    We call $X^*$ the \emph{completion} of $X$.
\end{qparts}
\end{question}

\section*{Other Problems}
\begin{question}[1]
Evaluate the following limits (in R) or show that the limit does not exist.
\begin{qparts}
  \item $\lim_{n\to\infty}(-1)^n$
  \item $\lim_{n\to\infty} \frac{n^2 - n}{n^3 + n}$
  \item $\lim_{n\to\infty} n^{1/n}$
\end{qparts}
\end{question}

\begin{question}[2]
  Let $(X, d)$ be a metric space, $\{p_n\} \subset X$ a convergent sequence with $p_n \to p$, and 
  $\{q_n\} \subset X$ a convergent sequence with $q_n \to q$. Prove that $d(p_n, q_n) \to d(p, q)$
  (This last convergence takes place in $\R$).
\end{question}

\end{document}
