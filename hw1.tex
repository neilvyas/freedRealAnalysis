\documentclass{assignment}
\usepackage{mymath}

\begin{document}
\header{Real Analysis 1}{Homework 1}

\question[1]{
  Prove that there is no rational number $x$ such that $x^2 = 3.$
}

\begin{proof}
Suppose not, that is, suppose that $\exists m,n\in\Z, n \neq 0 \st x = \frac{m}{n}$ and $x^2 = 3$,
where $a,b$ have no common factors. Then we have that $m^2 = 3n^2$. By the lemma, we have that since
$3\mid m^2$, $3\mid m$. Let $k\in\Z\st 3k = m$. So now we have that 
\begin{align*}
  m^2 = 3n^2 &\Rightarrow (3k)^2 = 3n^2 \\
             &\Rightarrow 9k^2 = 3n^2 \\
             &\Rightarrow 3k^2 = n^2,
\end{align*}
and by the lemma we have that, since $3\mid n^2$, $3\mid n$, which implies that $m$ and $n$ have a 
common factor - 3. This is a contradiction, since we supposed that $m,n$ had no common factors.
\end{proof}

\begin{lemma}
  For a prime $p$ and $a\in\Z$, if $p\mid a^2$, $p\mid a$.
\end{lemma}

\begin{proof}
  Consider the negation: $$\sim(p\mid a^2 \Rightarrow p\mid a) \cong p \nmid a^2 \wedge p\mid a.$$
  This is obviously false, since we have that $p\mid a \Rightarrow p\mid a^2$.
\end{proof}

\question[2]{
  \begin{qparts}
  \item Construct a field $F$ with two elements. 
  \item Construct a field $F$ with four elements.
  \item For these fields, given $a\in F$, can we always solve $x^3 = a$?
  \end{qparts}
}

\begin{qparts}
  \item This field is just the integers mod 2, or $\Z/2\Z.$ So our addition and multiplication tables
    are 
   \begin{center}
    \begin{tabular}{c|c|c}
      $+$ & \emph{0} & \emph{1} \\
      \hline 
      \emph{0} & 0 & 1 \\
      \emph{1} & 1 & 0 \\
    \end{tabular}

    \begin{tabular}{c|c|c}
      $\times$ & \emph{0} & \emph{1} \\
      \hline 
      \emph{0} & 0 & 0 \\
      \emph{1} & 0 & 1 \\
    \end{tabular}
  \end{center}

  \item We need to complete the field from (a), which we can do by considering the roots of the polynomial
    $x^2 + x + 1$; let's call one root of this polynomial $a$. Then note that $1 + a$ is also a solution,
    since we still have $1 + 1 = 0$, and 
    \begin{align*}
      (1 + a)^2 + (1 + a) + 1 &= 1^2 + a^2 + \bar{2}a + 1 + a + 1 \\
                              &= a^2 + a + 1.
    \end{align*}
    So then we can work out the tables, using the distributive property of a field, and the fact that 
    the non-zero elements of the field form a group; the only group with three elements is the cyclic 
    group $\Z/3\Z$. So we have
   \begin{center}
     \begin{tabular}{c|c|c|c|c}
       $+$ & \emph{0} & \emph{1} & \emph{$a$} & \emph{$1+a$} \\
      \hline 
      \emph{0} & 0 & 1 & $a$ & $1+a$ \\
      \emph{1} & 1 & 0 & $1 + a$ & $a$ \\
      \emph{$a$} & $a$ & $1+a$ & 0 & 1 \\
      \emph{$1+a$} & $1+a$ & $a$ & 1 & 0 
    \end{tabular}

    \begin{tabular}{c|c|c|c|c}
      $\times$ & \emph{0} & \emph{1} & \emph{$a$} & \emph{$1+a$} \\
      \hline 
      \emph{0} & 0 & 0 & 0 & 0 \\
      \emph{1} & 0 & 1 & $a$ & $1+a$ \\
      \emph{$a$} & 0 & $a$ & $1+a$ & 1 \\
      \emph{$1+a$} & 0 & $1+a$ & 1 & $a$ 
    \end{tabular}
  \end{center}

\item 
    For (a), we can always solve this equation, since $$0^3 = 0, 1^3 = 1.$$ 
    For (b), consider $x^2 = a$, where $a$ is the $a$ we defined and not the $a$ from the question. Note that
    we can't actually solve this, which we can check by plugging in every value for $x$: $$0^3 = 0,
    1^3 = 1, a^3 = 1, (1+a)^3 = 1.$$
\end{qparts}

\question[Rudin, 1.3]{
  Prove proposition 1.15.
}

\question[Proposition 1.15]{
  The axioms of multiplication imply the following statements.
  \begin{qparts}
  \item If $x\neq 0$ and $xy = xz$ then $y = z$.
  \item If $x\neq 0$ and $xy = x$ then $y = 1$.
  \item If $x\neq 0$ and $xy = 1$ then $y = \frac{1}{x}$.
  \item If $x\neq 0$ then $1/(1/x) = x$.
  \end{qparts}
}

\begin{proof}
  \vspace{1em}
  \begin{qparts}
  \item We have from (M5) that $x\neq 0 \Rightarrow \exists \frac{1}{x}\in F \st x(\frac{1}{x}) = 1$.
    So we simply multiply both sides on the left by $\frac{1}{x}$, which yields $y=z$.
  \item We invoke the exact same argument as above, noting that, here, we take $z=1$.
  \item We invoke the same arguent as above, again, but taking $z=\frac{1}{x}\in F$ this time.
  \item Say $b = \frac{1}{x}$; then we can restate the claim as $x\neq 0 \Rightarrow \frac{1}{b} = x$.
    Since $b\neq 0$, we have that $\frac{1}{b}\in F$ and $(\frac{1}{b})b = 1$, by (M5). Take the claim and multiply
    both sides by $b$, yielding $1 = xb$. By (a), we have that $$x = \frac{1}{b},$$ since $$1 = xb = \frac{1}{b}.$$
  \end{qparts}
\end{proof}

\end{document}
